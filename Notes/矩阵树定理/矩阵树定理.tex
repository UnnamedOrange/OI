\documentclass[UTF8]{article}
\usepackage{ctex}
\usepackage{hyperref} %用于设置 PDF 的信息
\usepackage{setspace} %用于设置行间距
\usepackage{listings} %用于代码高亮
\usepackage{xcolor} %用于处理颜色
\usepackage{ulem} %用于各种线
\usepackage{amsmath} %用于数学公式(如 \begin{align})
\usepackage{booktabs} %用于表格画线
\usepackage{graphicx} %用于插入图片
\usepackage[top = 1.0in, bottom = 1.0in, left = 1.0in, right = 1.0in]{geometry} %设置页边距

\hypersetup{
	pdfauthor={Orange}
}

\setmonofont{Consolas} %设置字体
\lstset{
    basicstyle = \normalsize\ttfamily,
    numbers = left,
    numberstyle = \tiny,
    keywordstyle = \color{blue!100},
    commentstyle = \color{red!50!green!50!blue!50},
    rulesepcolor = \color{red!20!green!20!blue!20},
    xleftmargin = 2em, xrightmargin = 2em, aboveskip = 1em,
}

\title{矩阵树定理}
\author{Orange}
\date{\today}

\begin{document}
	
	\yahei

	\maketitle

	\section{摘要}
	
	矩阵树定理(Matrix Tree Theorem)是用于解决
	带标号图的生成树计数问题的有力工具。
	本文从矩阵树定理的数学基础行列式出发,
	简单对其进行了证明,并对该类问题进行了简单扩展。

	\bigskip

	关键词:矩阵树定理~行列式

	\section{快速复习 —— 行列式}

	\subsection{矩阵与行列式}

	\subsubsection{矩阵}

	数学中,矩阵被定义为一个矩形的二维数组。
	一般地,我们用如下方式表示一个 $n$ 行 $m$ 列的矩阵:

	\begin{equation*}
		A = 
		\begin{bmatrix}
			a_{1, 1} & a_{1, 2} & \cdots & a_{1, m}
			\\
			a_{2, 1} & a_{2, 2} & \cdots & a_{2, m}
			\\
			\vdots & \vdots & \ddots & \vdots
			\\
			a_{n, 1} & a_{n, 2} & \cdots & a_{n, m}
		\end{bmatrix}
	\end{equation*}

	特别地,如果一个矩阵的行数和列数相等,即 $n = m$,
	那么我们称这个矩阵为\textbf{方阵}。

	\subsubsection{行列式}

	记:

	\begin{equation*}
		D = 
		\begin{vmatrix}
			a_{1, 1} & a_{1, 2} & \cdots & a_{1, n}
			\\
			a_{2, 1} & a_{2, 2} & \cdots & a_{2, n}
			\\
			\vdots & \vdots & \ddots & \vdots
			\\
			a_{n, 1} & a_{n, 2} & \cdots & a_{n, n}
		\end{vmatrix}
	\end{equation*}
	称为 \textbf{$n$ 阶行列式}。

	不同于矩阵,一个行列式表示\textbf{一个数},
	具体的运算法则我们将在后文提及。

	\bigskip

	对一个 $n \times n$ 的方阵 $A$,记:

	\begin{equation*}
		\det(A) = 
		\begin{vmatrix}
			a_{1, 1} & a_{1, 2} & \cdots & a_{1, n}
			\\
			a_{2, 1} & a_{2, 2} & \cdots & a_{2, n}
			\\
			\vdots & \vdots & \ddots & \vdots
			\\
			a_{n, 1} & a_{n, 2} & \cdots & a_{n, n}
		\end{vmatrix}
	\end{equation*}
	其中 $\det$ 是一个函数,表示{\bfseries 求矩阵 $A$ 相应的行列式}。

	\subsection{行列式的基本概念}

	为了方便,我们记 $D_n$ 表示一个一般的 $n$ 阶行列式,即:
	\begin{equation*}
		D_n = 
		\begin{vmatrix}
			a_{1, 1} & a_{1, 2} & \cdots & a_{1, n}
			\\
			a_{2, 1} & a_{2, 2} & \cdots & a_{2, n}
			\\
			\vdots & \vdots & \ddots & \vdots
			\\
			a_{n, 1} & a_{n, 2} & \cdots & a_{n, n}
		\end{vmatrix}
	\end{equation*}

	\subsubsection{排列与逆序数}

	我们称由 $1 \sim n$ 组成的一个有序数组
	$a_{1 \sim n}$ 为一个 \textbf{$n$ 级排列}。 
	显然,共有 $n!$ 个不同的 $n$ 级排列。

	在一个 $n$ 级排列中,若某两个数的前后位置与大小顺序不同,
	即大的在小的前面,则称这两个数构成一个\textbf{逆序对}。
	称排列 $a$ 中逆序对的总数为这个排列的\textbf{逆序数},
	记为 $N(a)$。

	如果一个排列的逆序数为奇数,我们称它为\textbf{奇排列};
	如果一个排列的逆序数为偶数,我们称它为\textbf{偶排列}。
	有一个不是很显然的结论:对于 $n!$ 个不同的 $n$ 级排列,
	奇排列和偶排列的个数分别为 $\frac {n!} {2}$,证明略。

	\subsubsection{对角线}

	我们称从左上角($a_{1, 1}$)到右下角($a_{n, n}$)为\textbf{主对角线};
	称从右上角($a_{1, n}$)到左下角($a_{n, 1}$)为\textbf{次对角线}。

	主对角线和次对角线各仅有一条。

	\subsubsection{计算行列式}

	我们定义 $n$ 阶行列式的运算结果为:

	\begin{equation*}
		D_n =
		\sum (-1)^{N(j_1 j_2 \cdots j_n)}
		(a_{1, j_1} \cdot a_{2, j_2} \cdot \cdots \cdot a_{n, j_n})
	\end{equation*}
	其中 $j_1 j_2 \cdots j_n$ 代表 $n$ 的一个排列,
	而求和表示遍历所有的 $n$ 级排列。

	\bigskip

	根据定义,我们有如下较为简单,但极为重要的结论。

	\leftline{\textcircled{\tiny{1}}
	下三角行列式的值等于其主对角线各元素的乘积:}

	\begin{equation*}
		\begin{vmatrix}
			a_{1, 1} & 0 & 0 & \cdots & 0
			\\
			a_{2, 1} & a_{2, 2} & 0 & \cdots & 0
			\\
			a_{3, 1} & a_{3, 2} & a_{3, 3} & \cdots & 0
			\\
			\vdots & \vdots & \vdots & \ddots & \vdots
			\\
			a_{n, 1} & a_{n, 2} & a_{n, 3} & \cdots & a_{n, n}
		\end{vmatrix}
		=
		\prod_{i = 1}^{n} a_{i, i}
	\end{equation*}

	\leftline{\textcircled{\tiny{2}}
	上三角行列式的值等于其主对角线各元素的乘积:}

	\begin{equation*}
		\begin{vmatrix}
			a_{1, 1} & a_{1, 2} & a_{1, 3} & \cdots & a_{1, n}
			\\
			0 & a_{2, 2} & a_{2, 3} & \cdots & a_{2, n}
			\\
			0 & 0 & a_{3, 3} & \cdots & a_{3, n}
			\\
			\vdots & \vdots & \vdots & \ddots & \vdots
			\\
			0 & 0 & 0 & \cdots & a_{n, n}
		\end{vmatrix}
		=
		\prod_{i = 1}^{n} a_{i, i}
	\end{equation*}

	\leftline{\textcircled{\tiny{3}}
	对角行列式的值等于其主对角线各元素的乘积:}

	\begin{equation*}
		\begin{vmatrix}
			a_{1, 1} & 0 & 0 & \cdots & 0
			\\
			0 & a_{2, 2} & 0 & \cdots & 0
			\\
			0 & 0 & a_{3, 3} & \cdots & 0
			\\
			\vdots & \vdots & \vdots & \ddots & \vdots
			\\
			0 & 0 & 0 & \cdots & a_{n, n}
		\end{vmatrix}
		=
		\prod_{i = 1}^{n} a_{i, i}
	\end{equation*}

	\bigskip

	类似地,对于除次对角线外的元素均为 $0$ 的行列式,
	它的值为:

	$$
	(-1)^{\frac {n(n - 1)} {2}} \prod_{i = 1}^{n} a_{i, n + 1 - i}
	$$

	\bigskip

	另外,对于一般的行列式 $D_n$,它也可以用如下的式子计算:
	\begin{equation*}
		D = \sum (-1)^{N(i_1 i_2 \cdots i_n) + N(j_1 j_2 \cdots j_n)}
		(a_{i_1, j_1} a_{i_2, j_2} \cdots a_{i_n, j_n})
	\end{equation*}
	其中 $i_1 i_2 \cdots i_n$ 为一个 $n$ 级排列,
	$j_1 j_2 \cdots j_n$ 为所有的 $n$ 级排列。

	\subsubsection{练习}

	\leftline{\textcircled{\tiny{1}}}

	\begin{align*}
		&
		\begin{vmatrix}
			0 & 1 & 0 & \cdots & 0
			\\
			0 & 0 & 2 & \cdots & 0
			\\
			\vdots & \vdots & \vdots & \ddots & \vdots
			\\
			0 & 0 & 0 & \cdots & n - 1
			\\
			n & 0 & 0 & \cdots & 0
		\end{vmatrix}
		\\=&
		(-1)^{N(2 \, 3 \, \cdots \, n \, 1)} (1 \times 2 \times \cdots \times n)
		\\=&
		(-1)^{n - 1} n!
	\end{align*}

	\leftline{\textcircled{\tiny{2}}}

	\begin{equation*}
		\begin{vmatrix}
		a_1 & b_1 & c_1 & d_1 & e_1
		\\
		a_2 & b_2 & c_2 & d_2 & e_2
		\\
		a_3 & b_3 & 0 & 0 & 0
		\\
		a_4 & b_4 & 0 & 0 & 0
		\\
		a_4 & b_5 & 0 & 0 & 0
		\end{vmatrix}
		= 0
	\end{equation*}

	对于行列式的每一项,无法避免至少有一个数为 $0$,所以行列式为 $0$。

	\subsection{行列式初等变换}

	我们可以利用行列式的定义直接计算行列式,
	但是这么做的时间复杂度是指数级的,
	没有应用价值。
	如果我们有一个对角行列式,那么我们就能在线性时间复杂度内计算它。

	将一个一般的行列式转变成等效的对角行列式的方法就是
	利用\textbf{行列式的性质}进行\textbf{行列式初等变换}。

	\subsubsection{性质}

	性质 1
	
	\qquad $D_n = D_n^T$,其中 $D_n^T$ 为 $D_n$ 的转置。

	\bigskip

	性质 2
	
	\qquad 互换行列式的两行或两列,行列式的值变号。
	
	\qquad 推论:若 $D_n$ 的两行或两列完全相同,则 $D_n = 0$。

	\bigskip
	
	性质 3
	
	\qquad 用数 $k$ 乘以行列式的某一行或某一列,等于 $k$ 乘以行列式。
	
	\qquad 推论 1:$D$ 中某一行或某一列中所有元素的因子可以提到行列式的外面。
	
	\qquad 推论 2:若 $D$ 的某两行或某两列的对应元素成比例,则 $D = 0$。
	
	\bigskip

	性质 4
	
	\qquad 行列式 $D$ 的某一行或某一列的所有元素都乘以数 $k$
	加到另一行或另一列(行对行,列对列)的相应元素上,行列式的值不变。

	\subsection{利用性质对行列式进行``高斯消元''}
	
	观察性质 4,可以发现,我们完全可以仿照高斯消元处理行列式的方法,
	将行列式变成一个上三角矩阵。可以这样做的原因是
	\textbf{矩阵初等变换与行列式初等变换是类似的}。
	借此我们得到了一个时间复杂度为 $O(n^3)$ 的计算行列式的值的算法:

	\lstset{language=C++}
	\begin{lstlisting}
int main() { return 0; }
	\end{lstlisting}
	
	\subsection{余子式}

	\section{矩阵树定理}
	
	一般地,我们认为矩阵树定理用于解决
	\textbf{带标号无向图}的生成树计数问题。

	\subsection{三个重要矩阵}

	如不加说明,我们认为我们的图有 $n$ 个点,$m$ 条边。

	\subsubsection{图 $G$ 的度数矩阵 $D$}

	我们定义图 $G$ 的度数矩阵 $D$ 是一个 $n \times n$ 的矩阵,
	满足:
	
	$$
	D_{i, j} = 0 \pod {i \ne j}
	$$$$
	D_{i, i} = d_i
	$$

	其中 $d_i$ 表示点 $i$ 的度数。

	\subsubsection{图 $G$ 的邻接矩阵 $A$}

	图 $G$ 的邻接矩阵 $A$ 也是一个 $n \times n$ 的矩阵,
	满足:

	$$
	A_{i, i} = 0
	$$$$
	A_{i, j} = A_{j, i} = a_{i, j}
	$$

	其中 $a_{i, j}$ 表示 $i, j$ 之间的边数。
	如果保证不存在重边,
	那么这个矩阵一定是一个仅含有 $0$ 和 $1$ 的矩阵。

	\subsubsection{基尔霍夫矩阵}

	我们定义一张图的\textbf{基尔霍夫(Kirchhoff)矩阵 K}
	为一个 $n \times n$ 的矩阵,
	满足:

	$$
	K = D - A
	$$

	我们又称基尔霍夫矩阵 $K$ 为\textbf{拉普拉斯算子}。

	

\end{document}