\documentclass[UTF8]{article}
\usepackage{ctex}
\usepackage{hyperref} %用于设置 PDF 的信息
\usepackage{setspace} %用于设置行间距
\usepackage{listings} %用于代码高亮
\usepackage{xcolor} %用于处理颜色
\usepackage{ulem} %用于各种线
\usepackage{amsmath} %用于数学公式(如 \begin{align})
\usepackage{booktabs} %用于表格画线
\usepackage{graphicx} %用于插入图片
\usepackage[top = 1.0in, bottom = 1.0in, left = 1.0in, right = 1.0in]{geometry} %设置页边距

\hypersetup{
	pdfauthor={Orange}
}

\title{Solution}
\author{Orange}
\date{\today}

\newcommand{\floorfrac}[2]{\left\lfloor \frac {#1} {#2} \right\rfloor}

\begin{document}
	\heiti
	%\maketitle

	\section{GCD 6}

	\subsection{第一步}

	$$
	\sum_{x = 1}^{n} \sum_{y = 1}^{n} \mathrm{lcm}(x, y)
	$$

	设 $f(n) = \sum_{i = 1}^{n} \mathrm{lcm}(i, n)$,则原答案为:
	$$
	\sum_{i = 1}^{n} (2f(i) - i)
	$$

	\subsection{第二步}

	\begin{align*}
		f(n) =&
		\sum_{x = 1}^{n} \mathrm{lcm}(x, n)
		\\=&
		\sum_{x = 1}^{n} \frac {xn} {\gcd(x, n)}
		\\=&
		n
		\sum_{g \mid n} \frac {1} {g}
		\sum_{x = 1}^{\frac {n} {g}} [\gcd(x, \frac {n} {g}) = 1] (xg)
		\\=&
		n
		\sum_{g \mid n}
		\sum_{x = 1}^{\frac {n} {g}} [\gcd(x, \frac {n} {g}) = 1] x
	\end{align*}

	等价于:
	$$
	= n
	\sum_{g \mid n}
	\sum_{x = 1}^{g} [\gcd(x, g) = 1] x
	$$

	如何计算?既然能够想到 $\frac {n} {g}$ 等价于 $g$,
	那也应该不难想到若 $\gcd(x, g) = 1(x < g)$,则 $\gcd(g - x, g) = 1$。
	不难发现其否命题也成立。我们将 $x$ 与 $g - x$ 合起来处理,则有:
	$$
	\sum_{x = 1}^{g} [\gcd(x, g) = 1] x
	= \frac {1} {2} g \cdot \varphi(g)
	$$

	特别地,当 $g = 1$ 时,上式为 $1$。
	所以我们只需要算出:
	$$
	h(n) = \sum_{g \mid n} g \cdot \varphi(g)
	$$
	即可。

	即:

	$$
	(\mathrm {id} \cdot \varphi) * 1
	$$

	根据积性函数的乘积是积性函数,积性函数的卷积是积性函数,
	我们便知道了 $h$ 可以使用线性筛预处理。

	\bigskip
	我们有:
	$$
	h(1) = 1
	$$
	$$
	h(p) = p \cdot (p - 1) + 1
	$$
	$$
	h(p^k) = (p + p^3 + p^5 + \cdots + p^{2k - 1})(p - 1) + 1
	$$

	如果直接递推,会很复杂,时间复杂度也难以得到保证,所以考虑继续化简:
	\begin{align*}
	h(p^k) =& (p + p^3 + p^5 + \cdots + p^{2k - 1})(p - 1) + 1
	\\=&
	\frac {p^{2k + 1} + 1} {p + 1}
	\end{align*}

	可以知道:
	\begin{align*}
	h(p^{k + 1}) =& \frac {p^{2k + 3} + p^2 - p^2 + 1} {p + 1}
	\\=&
	h(p^k) \cdot p^2 - (p - 1)
	\end{align*}

	这样,通过维护最小因子出现次数,
	便能在 $O(1)$ 的时间复杂度内递推出 $h(p^{k + 1})$。

	\subsection{具体实现}

	一步一步代回:
	\begin{align*}&
		2f(n) - n
		\\=&
		2(\frac {1} {2} n \cdot (h(n) + 1)) - n
	\end{align*}

	加 1 是为了统一 $h(1) = 1$ 的情况。

	\begin{align*}
		=&
		n \cdot (h(n) + 1) - n
		\\=&
		n \cdot h(n)
	\end{align*}

	这样问题就很简单了。
\end{document}