\documentclass[UTF8]{article}
\usepackage{ctex}
\usepackage{hyperref} %用于设置 PDF 的信息
\usepackage{setspace} %用于设置行间距
\usepackage{listings} %用于代码高亮
\usepackage{xcolor} %用于处理颜色
\usepackage{ulem} %用于各种线
\usepackage{amsmath} %用于数学公式(如 \begin{align})
\usepackage{booktabs} %用于表格画线
\usepackage{graphicx} %用于插入图片
\usepackage[top = 1.0in, bottom = 1.0in, left = 1.0in, right = 1.0in]{geometry} %设置页边距

\hypersetup{
	pdfauthor={Orange}
}

\title{Prob. 1}
\author{Orange}
\date{\today}

\begin{document}
	\heiti

	\section{Prob. 1 \small {prob1.cpp / prob1.in / prob1.out} TL: 1s ML: 512 MiB}

	\subsection{问题描述}

	给你一个仅由小写字母组成的非空字符串 $s$,
	希望你能求出它有多少个出现次数恰好等于 $k$ 次的子串。

	\subsection{输入}

	输入包含两行。

	第一行输入一行一个由小写字母组成的字符串 $s$。

	第二行输入一行一个正整数 $k$。

	\subsection{输出}

	输出包含一行一个正整数表示答案。

	\subsection{样例}

	\subsubsection{输入}

	ababaab

	3

	\subsubsection{输出}

	2

	\subsubsection{解释}

	子串 ``b'',``ab'' 恰好出现了 3 次。

	\subsection{数据规模与约定}

	对于 $30\%$ 的数据,$|S| \le 10^3$。

	对于 $100\%$ 的数据,$|S| \le 2 \times 10^5$,$1 \le k \le |S|$。

\end{document}