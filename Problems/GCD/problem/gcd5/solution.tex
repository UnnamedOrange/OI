\documentclass[12pt, UTF8]{article}
\usepackage{ctex}
\usepackage{hyperref} %用于设置 PDF 的信息
\usepackage{setspace} %用于设置行间距
\usepackage{listings} %用于代码高亮 
\usepackage{xcolor} %用于处理颜色
\usepackage{ulem} %用于各种线
\usepackage{amsmath} %用于数学公式(如 \begin{align})
\usepackage{booktabs} %用于表格画线
\usepackage{graphicx} %用于插入图片
\usepackage[top = 1.0in, bottom = 1.0in, left = 1.0in, right = 1.0in]{geometry} %设置页边距

\hypersetup{
	pdfauthor={Orange}
}

\title{Solution}
\author{Orange}
\date{\today}

\newcommand{\floorfrac}[2]{\left\lfloor \frac {#1} {#2} \right\rfloor}

\begin{document}
	\heiti
	%\maketitle

	\section{GCD 5}

	\subsection{60 分做法}

	\begin{align*}
		& \sum_{x = 1}^{n} \sum_{y = 1}^{m} \mathrm{lcm}(x, y)
		\\
		=& \sum_{x = 1}^{n} \sum_{y = 1}^{m} \frac {xy} {\gcd(x, y)}
		\\
		=& \sum_{g = 1}
		\frac {1} {g}
		\sum_{x = 1}^{\floorfrac {n} {g}}
		\sum_{y = 1}^{\floorfrac {m} {g}}
		[\gcd(x, y) = 1]
		(xg)(yg)
		\\
		=& \sum_{g = 1} g
		\sum_{x = 1}^{\floorfrac {n} {g}}
		\sum_{y = 1}^{\floorfrac {m} {g}}
		xy
		\sum_{d \mid \gcd(x, y)} \mu(d)
		\\
		=& \sum_{g = 1} g
		\sum_{d = 1} \mu(d)
		\sum_{x = 1}^{\floorfrac {n} {gd}}
		\sum_{y = 1}^{\floorfrac {m} {gd}}
		(xd)(yd)
		\\
		=& \sum_{g = 1} g
		\sum_{d = 1} d^2 \mu(d)
		\sum_{x = 1}^{\floorfrac {n} {gd}}
		\sum_{y = 1}^{\floorfrac {m} {gd}}
		xy
		\\
		=& \sum_{g = 1} g
		\sum_{d = 1} d^2 \mu(d)
		(\sum_{x = 1}^{\floorfrac {n} {gd}} x)
		(\sum_{y = 1}^{\floorfrac {m} {gd}} y)
		\\
		=& \frac {1} {4}
		\sum_{g = 1} g
		\sum_{d = 1} d^2 \mu(d)
		\floorfrac {n} {gd}
		(\floorfrac {n} {gd} + 1)
		\floorfrac {m} {gd}
		(\floorfrac {m} {gd} + 1)
	\end{align*}

	可以知道:
	$$
	F(n) = n^2 \mu(n)
	$$
	是积性函数,预处理后加上分块能够得到 60 分。

	\subsection{100 分做法}

	\bigskip
	继续化简:
	\begin{align*}
		=& \frac {1} {4}
		\sum_{t = 1}
		\floorfrac {n} {t}
		(\floorfrac {n} {t} + 1)
		\floorfrac {m} {t}
		(\floorfrac {m} {t} + 1)
		\sum_{g \mid t}
		g \cdot (\frac {t} {g})^2 \mu(\frac {t} {g})
	\end{align*}
	
	右边的和式是一个狄利克雷卷积的形式,且原函数均为积性函数,
	所以可以使用欧拉筛。

	设:
	$$
	G(n) = \sum_{d \mid n} d \cdot (\frac {n} {d})^2 \mu(\frac {n} {d}) 
	$$

	可以知道:
	$$
	G(n) = 1
	$$
	$$
	G(p) = p - p^2
	$$
	$$
	G(p^k) = p^k - p^{k + 1}
	$$
	$$
	G(p^{k + 1}) = pG(p^k)
	$$

\end{document}