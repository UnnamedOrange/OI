\documentclass[12pt, hyperref, UTF8]{article}
\usepackage{ctex}
\usepackage{hyperref} %用于设置 PDF 的信息
\usepackage{setspace} %用于设置行间距
\usepackage{listings} %用于代码高亮 
\usepackage{xcolor}
\usepackage{ulem} %用于各种线
\usepackage[top = 1.0in, bottom = 1.0in, left = 1.0in, right = 1.0in]{geometry} %设置页边距

\hypersetup{
	pdfauthor={Orange}
}

\title{Solution}
\author{Orange}
\date{\today}

\begin{document}
	\heiti
	\maketitle

	\begin{center}
		由于题很水,所以题解会更水。
	\end{center}

	\newpage
	\section{Wander}
	\subsection{20~分做法}
	这个~20~分指拿到了~50\%~的数据的第一问的分。

	求最短路就好了。

	\subsection{50~分做法}
	指拿到了~50\%~的数据的所有分。
	
	这个做法很经典,很重要。
	先做两次最短路:以~1~为起点的单源最短路,
	和以~n~为起点\textbf{倒着走}的单源最短路。
	实现方式很简单,输入时额外保存一下\textbf{反图}即可。
	反图就是把所有边反向的图。

	现在我们知道了\textbf{从~1~出发到任意点的最短路和从任意点出发到~n~的最短路}。

	然后\textbf{枚举让哪条边的权值变为~0。}如果我们不经过这条权值变为~0~的边,
	答案一定不会比第一问的答案优,所以我们一定会经过这条边。
	如果把这条边的权值变为了~0,就相当于是我们从~1~走到了这条边的起点,
	再从这条边的终点走到了~n~。利用之前求的两个最短路即可得出答案。

	\subsection{100~分做法}
	超时了?难道最短路写错了?
	
	Hint~都提示了要注意(最短路的)时间复杂度。什么,你忘了?
	那我在这里再讲讲。

	原始的~Bellman-Ford~算法,通过对所有边进行~n - 1~次松弛操作,
	能在~$O(nm)$~的时间复杂度内求得最短路。
	这个算法有个优化:在一整次松弛中,如果所有边都不能再被松弛,
	便可以直接退出,类似于冒泡排序中的那个优化。但是时间复杂度不会改变。

	\bigskip
	使用队列改进后的~Bellman-Ford~算法,即熟知的~SPFA~算法,
	每次只将当前可能发生松弛操作的边的起点入队,对于绝大部分的图,
	SPFA~算法都能有良好的表现。

	不少\textbf{过去}的书都说~SPFA~算法的\textbf{平均}时间复杂度为
	~$O(kE)$,其中~$E$~为边数,$k$为小常数,一般~$\le 2$。

	\Large
	然而以上都是胡扯,SPFA~算法的最坏时间复杂度为~$O(nm)$,
	而且这种数据相当好造……

	\normalsize
	虽然在平均情况下,SPFA~表现优秀,但是在非常状况下,
	\textbf{如果不存在负权边},不建议用~SPFA~算法。
	
	\bigskip
	那用什么算法呢?当然是~Dijkstra~算法啦。

	具体内容我不想讲了,你们应该都会,
	只不过你们可能只会时间复杂度为~$O(n^2)$~的算法。

	紫书上对时间复杂度为~$O(m \log n)$~的~Dijkstra~算法有详细的讲解
	并给出了完整代码。
	看完后,我只想补充一下为什么时间复杂度为~$O(m \log n)$~而不是
	~$O(m \log m)$。

	\bigskip
	在最坏情况下,给定的会是一张完全图,边数~$m = n^2$。

	\begin{center}
		$O(m \log m) = O(2 m \log n) = O(m \log n)$
	\end{center}

	\subsection{后话}
	应该没有爆零的同学吧?为了送出分数,这道题我特地用~Special Judge~来测的。
	只不过你们现在知道这个模型了,以后我就不对这个模型给部分分了。

	如果真的有爆零的同学,那就太不应该了:只输出两个~0~都能得到至少~5~分。

	\bigskip
	不知道有没有同学用贪心做的第二问。可以试试将最短路上边权最大的边设为~0。
	虽然肯定是错的,但说不定能多拿三分。\textbf{别忘了三分也是分。}

	\section{Gene}

	\subsection{前言}
	我只是想考考你们排序,怎么写排序函数我都讲过。

	\subsection{80~分做法}
	直接使用间接排序即可。

	\subsection{100~分做法}
	开一个结构体,结构体内保存序列以及其起始的序号。
	排序时,除了按照以序列为第一关键字排序,还要以序号为第二关键字排序,
	这样就保证了相同序列中一开始序号小的排在前面。

	\subsection{std~的做法}
	还是使用间接排序。然后用一个叫做~\textbf{stable\_sort}~的东西。
	
	\bigskip
	这个就是所谓的\textbf{稳定排序}。
	
	稳定排序并不是指时间复杂度稳定,而是指对于“相等”的两个元素,
	在排序前后相对位置不变,这道题就是个很好的例子。
	这里的“相等”指满足这种关系的两种元素:
	\begin{center}
		$!(u < v) ~\&~ !(v < u)$
	\end{center}
	
	\textbf{对于排序的比较函数,如果两个元素满足了上述关系,
	必须认为这两个元素相等。
	换句话说,如果对比较函数满足了上式,而我们却认为这两个元素不相等,
	那么这个比较函数不能作为排序函数的比较函数。}

	对这个问题的详细讨论已经超出了这个地方的范围。
	\sout{说不定上面都是我乱编的},
	所以请利用好网络资源,自行了解相关内容。

	\bigskip
	我能保证是正确的是,\textbf{快速排序是不稳定排序,归并排序是稳定排序。}
	换句话说,基于快速排序的~sort~是不稳定排序,
	而基于归并排序的~stable\_sort~是稳定排序。

	\subsection{后记}
	由于我电脑很快,所以我感觉卡常是不可能的了。
	如果你们在学校的电脑上被卡了,很有可能这个时候~std~也是个~TLE。

	\section{Joseph}

	\subsection{30~分做法}
	用数组模拟。

	\subsection{60~分做法}
	用链表模拟。

	\subsection{80~分做法}
	用动态规划(递推)。
	
	看蓝书,写得相当清楚。

	\subsection{另外~20~分做法}
	这~20~分其实是一道初赛题。
	初赛模拟时~ztx~就做出来了,所以我相信你们是可以想出来的,
	\textbf{毕竟时间是相当充足的}。

	你们去问~ztx~吧。

	\subsection{后记}
	Presentation~第~54~页叫你们思考过了。
	就算没有思考出来,我相信你们还是能拿~60~分吧。

\end{document}