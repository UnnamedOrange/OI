\documentclass[12pt, hyperref, UTF8]{article}
\usepackage{ctex}
\usepackage{hyperref} %用于设置 PDF 的信息
\usepackage{setspace} %用于设置行间距
\usepackage{listings} %用于代码高亮 
\usepackage{xcolor}
\usepackage{ulem} %用于各种线
\usepackage[top = 1.0in, bottom = 1.0in, left = 1.0in, right = 1.0in]{geometry} %设置页边距

\hypersetup{
	pdfauthor={Orange}
}
\title{基础题}
\author{Orange}
\date{\today}

\begin{document}
	\heiti

	\maketitle
	\bigskip
	\begin{center}
		\Large{所有测试均\textbf{不}打开~-O2~编译选项}

		\Large{所有测试均\textbf{不}打开~-std=c++11~编译选项}

		\vspace{15mm}
		\Large{题目难度与顺序无关}
		
		\Large{数据很弱,欢迎水过}
	\end{center}
	
	\newpage
	\section{Wander \small{wander.cpp/wander.in/wander.out
	(TL: 1s, ML: 512 MiB)}}

	\subsection{Description}
	众而周知,2018-1-31 是个神奇的日子。这天晚上,不仅有百年一遇的蓝月,
	还有千年一遇的奇观 ~ —— ~ wyf 梦游。
	
	具体地说,wyf 梦见自己正在一张奇怪的图中。
	这张图有~$n$~个结点,$m$~条 \textbf{带权有向边}。
	此时,wyf 正在结点~$1$,而教室竟然在遥远的结点~$n$!
	
	wyf~自然是很害怕迟到的,于是他想知道从~$1$~到~$n$~的最短路有多长。

	想必大家是知道的,做梦时人经常会闪现。
	wyf~也是这样:虽然他不能梦游时在现实世界中闪现,
	但是在梦境中闪现一下,wyf~也是很在行的。

	于是他\textbf{还}想知道:
	\underline{若能够将任意一条边的边权变为~0},从~$1$~到~$n$~的最短路有多长。

	\subsection{Input}
	第一行输入两个整数~$n$~和~$m$~代表点数和边数。

	接下来的~$m$~行,每行输入三个整数,代表一条边的起点,终点和边权。

	\subsection{Output}
	输出包含两个整数。

	第一个整数,代表在原图中~$1$~到~$n$~的最短路长度。

	第二个整数,代表在能够将任意一条边的边权变为~$0$~的条件下,$1$~到~$n$~的最短路。

	\subsection{Special Judge}
	这道题使用~Special Judge。
	
	如果你正确输出了第一个答案,你将获得~$40\%$~的分数。

	如果你正确输出了第二个答案,你将获得~$60\%$~的分数。

	如果你不知道某个答案,请最好输出~$0$。
	这不仅可以防止错误判断你的正确答案,也能增加你在本题中得分的几率。

	\newpage
	\subsection{Sample Input}
	3 3
	
	1 2 5
	
	2 3 4
	
	1 3 10

	\subsection{Sample Output}
	9

	0

	\subsection{Range}
	对于~$5\%$~的数据,wfy~就在教室里。
	
	对于另外~$5\%$~的数据,$n, m \le 10$。

	对于另外~$20\%$~的数据,$n \le 400$,$m \le 600$。

	对于另外~$20\%$~的数据,数据保证随机。

	对于~$100\%$~的数据,$n, m \le 200000$,
	\underline{边权~$\le 10^9$~且为正整数。}

	可能有重边和自环~——~但是这并不影响聪明的你解决这个问题。

	\subsection{Hint}
	请注意输入输出时程序的效率。

	请注意数据类型。

	如果不明白~Special Judge~是什么意思,你可以理解为这道题有部分分。

	因为没有开启~-O2~编译选项,所以请注意自己程序的效率。

	请注意自己程序的时间复杂度。

	\newpage
	\section{Gene \small{gene.cpp/gene.in/gene.out
	(TL: 1s, ML: 512 MiB)}}

	\subsection{Description}
	众而周知,yzm~是一位生竞大神。这天,yzm~正在研究一堆神奇的基因序列,
	然而他被巨大的计算量吓住了,所以他决定向你求助。

	一个基因序列可以抽象为一个\textbf{包含有~$c_i$~个数的正整数序列。}
	yzm~的任务是,把给定的~$n$~个序列按一定顺序排序。
	yzm~是个有强迫症的人,为了好看,他决定\textbf{以长度为第一关键字,从小到大排序。}
	形式化地,若排好序后给每个序列重新从~1~到~n~编号,则有~$c_1 \le c_2 \le ... \le c_n$。

	对于长度一样的序列,yzm~决定按字典序\textbf{从大到小}进行排序。
	形式化地,若排好序后给每个序列重新从~1~到~n~编号,对于长度相同的序列,
	有~$d_{i} \ge d_{i + 1} \ge d_{i + 2} \ge ... \ge d_{i + k}$,
	其中~$d_i$~代表第~i~个序列的字典序。

	yzm~并不想知道这堆序列最后长什么样子。他决定将原始的序列从~$1$~到~$n$~编号,然后一一告诉你。
	你的任务是告诉他最终的编号。
	形式化地,设最后的答案为~$t_i$,设~$<$~为按以上要求比较的小于号,
	则有~$a_{t_1} < a_{t_2} < a_{t_3} < ... < a_{t_n}$。
	特别地,\textbf{如果原来有两个相同的序列~$a_i$~和~$a_j$ $(i < j)$,
	那么在输出时~$i$~应该在~$j$~之前输出。}

	\subsection{Input}
	第一行输入一个整数~$n$,代表有~$n$~个序列。

	接下来的~$n$~行,每行首先输入一个整数~$c_i$,代表第~$i$~个序列有~$c_i$~个数。
	接下来输入~$c_i$~个整数,表示第~$i$~个序列的~$c_i$~个整数~$a_{i, j}$。

	\subsection{Output}
	输出包含一行~$n$~个整数,代表题目中要求输出的编号。
	
	\newpage
	\subsection{Sample Input}
	10

	5 4 4 6 10 10
	
	5 10 3 3 6 4
	
	2 7 5
	
	5 5 10 10 7 4
	
	3 10 9 3
	
	1 1
	
	2 10 1
	
	1 9
	
	3 2 7 8

	1 9

	\subsection{Sample Output}
	8 10 6 7 3 5 9 2 4 1

	\subsection{Range}
	对于~$10\%$~的数据,$n \le 10$,$\max \{ c_i \} \approx 5$。

	对于~$30\%$~的数据,$n \le 10^3$,$\max\{ c_i \} \approx 10^2$,
	$c_i = c_j (i \ne j)$。

	对于~$50\%$~的数据,$n \le 10^3$,$\sum c_i \approx 10^5$。
	
	对于~$80\%$~的数据,保证不会出现相同的序列。

	对于~$100\%$~的数据,$n \le 10^4$,$\sum c_i \approx 10^6$,$|a_{i, j}| \le 10^9$。

	\subsection{Hint}
	约等号是什么意思呢?就是叫你用~vector~的意思。

	\newpage
	\section{Joseph \small{joseph.cpp/joseph.in/joseph.out
	(TL: 1s, ML: 512 MiB)}}

	\subsection{Description}
	众而周知,约瑟夫问题是一个经典的问题,这道题就是要你解决经典的约瑟夫问题。

	\bigskip

	有~$n$~只野生~zys~站成一圈,标号为~$1$~到~$n$。
	编号为~$1$~的~zys~从~$1$~开始报~$1$,然后编号为~$2$~的~zys~报~$2$……
	当报数报到~$m$~的时候,报~$m$~的~zys~就离开这个圈,
	然后他的下一个~zys~开始报~$1$……

	最后整个圈就只剩下一只~zys~了。请你输出他的编号。

	\subsection{Input}
	输入包含一行两个正整数~$n$,$m$。

	\subsection{Output}
	输出包含一行一个正整数,代表最后一只~zys~的编号。

	%\newpage
	\subsection{Sample Input}
	5 3

	\subsection{Sample Output}
	4

	\subsection{Range}
	对于~$30\%$~的数据,$n \le 10^2$,$nm \le 10^5$。

	对于~$60\%$~的数据,$nm \le 10^7$。
	
	对于~$80\%$~的数据,$n \le 10^7$,$m \le 10^9$。
	
	对于另外~$20\%$~的数据,$n \le 10^9$, $m = 2$。

\end{document}