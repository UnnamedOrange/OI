\documentclass[UTF8]{article}
\usepackage{ctex}
\usepackage{hyperref} %用于设置 PDF 的信息
\usepackage{setspace} %用于设置行间距
\usepackage{listings} %用于代码高亮
\usepackage{xcolor} %用于处理颜色
\usepackage{ulem} %用于各种线
\usepackage{amsmath} %用于数学公式(如 \begin{align})
\usepackage{booktabs} %用于表格画线
\usepackage{graphicx} %用于插入图片
\usepackage[top = 1.0in, bottom = 1.0in, left = 1.0in, right = 1.0in]{geometry} %设置页边距

\hypersetup{
	pdfauthor={Orange}
}

\title{Solution}
\author{Orange}
\date{\today}

\newcommand{\floorfrac}[2]{\left\lfloor \frac {#1} {#2} \right\rfloor}

\begin{document}
	\heiti
	%\maketitle

	\section{GCD 7}

	\subsection{第一步}

	\begin{align*}&
		\sum_{x = 1}^{n} \sum_{y = 1}^{m} d(xy)
		\\=&
		\sum_{x = 1}^{n} \sum_{y = 1}^{m}
		\sum_{i \mid x} \sum_{j \mid y} [\gcd(i, j) = 1]
	\end{align*}

	\leftline {证明:}

	设
	$$
	xy =
	p_{1}^{r_{x_1} + r_{y_1}} \times
	p_{2}^{r_{x_2} + r_{y_2}} \times
	\cdots \times
	p_{k}^{r_{x_k} + r_{y_k}}
	$$

	有:
	$$
	d(xy) =
	(r_{x_1} + r_{y_1} + 1) \times
	(r_{x_2} + r_{y_2} + 1) \times
	\cdots \times
	(r_{x_k} + r_{y_k} + 1)
	$$

	若要有 $\gcd(x, y) = 1$,则对于某个质因数 $p_i$,
	必有 $r_{x_i} = 0$ 或 $r_{y_i} = 0$,或两者同时满足。
	依次考虑每个质因数:若 $r_{x_i} \ne 0$,则有 $r_{x_i}$ 种情况;
	若 $r_{y_i} \ne 0$,则有 $r_{y_i}$ 种情况;
	都为 $0$ 时只有一种情况。
	当我们使用乘法原理将各情况合并起来时,
	原命题得证。

	\bigskip
	为什么会想到这一步转换,我也不知道。考虑到这道题的重点还是莫比乌斯反演,
	所以就暂时不要纠结为什么第一步会想到这个了。

	\bigskip\bigskip\bigskip
	\leftline{更新:想到的原因}

	设 $x = p_1^{x_1} p_2^{x_2} \cdots p_k^{x_k}$,
	$y = p_1^{y_1} p_2^{y_2} \cdots p_k^{y_k}$。

	令 $d_1 = p_1^{z_{11}} p_2^{z_{12}} \cdots p_k^{z_{1k}}$,
	$d_2 = p_1^{z_{21}} p_2^{z_{22}} \cdots p_k^{z_{2k}}$,
	当 $z_{1i} + z_{2i} < x_i$ 时,我们强制让 $z_{2i} = 0$;
	当 $x_i \le z_{1i} + z_{2i} \le x_i + y_i$ 时,
	我们强制让 $z_{1i} = x_i$。
	则 $xy$ 的所有因数都可以使用 $d_1 d_2$ 不重不漏地表示出来,一一对应。
	
	\bigskip
	由定义必有:
	\begin{gather*}
		d_1 \mid x
		\\
		d_2 \mid y
	\end{gather*}
	
	同时可知:
	$$
	\gcd(\frac {x} {d_1} , d_2) = 1
	$$
	否则不符合我们的定义。

	则 $xy$ 的因数个数与以下情况一一对应:
	$$
	\sum_{\frac {x} {d_1} \mid x} \sum_{d_2 \mid y} [\gcd(\frac {x} {d_1}, d_2) = 1]
	$$

	即:
	$$
	\sum_{i \mid x} \sum_{j \mid y} [\gcd(i, j) = 1]
	$$

	\subsection{第二步}

	\begin{align*}&
		\sum_{x = 1}^{n} \sum_{y = 1}^{m}
		\sum_{i \mid x} \sum_{j \mid y} [\gcd(i, j) = 1]
		\\=&
		\sum_{x = 1}^{n} \sum_{y = 1}^{m}
		\sum_{i \mid x} \sum_{j \mid y}
		\sum_{d \mid \gcd(i, j)} \mu(d)
	\end{align*}

	众所周知,$d \mid \gcd(i, j)$ 等价于 $d \mid i$ 且 $d \mid j$。

	\begin{align*}=&
		\sum_{d = 1} \mu(d)
		\sum_{i = 1}^{\floorfrac nd}
		\sum_{j = 1}^{\floorfrac md}
		\sum_{x = 1, i \mid x}^{\floorfrac nd}
		\sum_{y = 1, j \mid y}^{\floorfrac md}
		1
	\end{align*}

	上一步的意思是,在最外面枚举 $d$,然后从内到外考虑贡献。
	考虑贡献时,首先考虑 $i$ 和 $j$,再结合 $i$ 和 $j$ 考虑 $x$ 和 $y$。

	\begin{align*}=&
		\sum_{d = 1} \mu(d)
		\sum_{i = 1}^{\floorfrac nd}
		\sum_{j = 1}^{\floorfrac md}
		\sum_{x = 1}^{\floorfrac n {id}}
		\sum_{y = 1}^{\floorfrac m {jd}}
		1
		\\=&
		\sum_{d = 1} \mu(d)
		\sum_{i = 1}^{\floorfrac nd}
		\sum_{j = 1}^{\floorfrac md}
		\left(
		\left( \sum_{x = 1}^{\floorfrac n {id}} 1 \right)
		\left( \sum_{y = 1}^{\floorfrac m {jd}} 1 \right)
		\right)
		\\=&
		\sum_{d = 1} \mu(d)
		\left(
		\sum_{i = 1}^{\floorfrac nd}
		\sum_{x = 1}^{\floorfrac n {id}} 1
		\right)
		\left(
		\sum_{j = 1}^{\floorfrac md}
		\sum_{y = 1}^{\floorfrac m {jd}} 1
		\right)
	\end{align*}

	以上变换可以有一般性的证明。

	$$
	= \sum_{d = 1} \mu(d)
	\left(
	\sum_{i = 1}^{\floorfrac nd} \floorfrac {\floorfrac n d} i
	\right)
	\left(
	\sum_{j = 1}^{\floorfrac md} \floorfrac {\floorfrac m d} i
	\right)
	$$

	\bigskip
	用 $O(n \sqrt n)$ 的时间复杂度内预处理
	$$
	\sum_{i = 1}^{n} \frac {n} {i}
	$$

	单次查询的时间复杂度为 $O(\sqrt n)$。
\end{document}