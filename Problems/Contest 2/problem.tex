\documentclass[UTF8]{article}
\usepackage{ctex}
\usepackage{hyperref} %用于设置 PDF 的信息
\usepackage{setspace} %用于设置行间距
\usepackage{listings} %用于代码高亮 
\usepackage{xcolor} %用于处理颜色
\usepackage{ulem} %用于各种线
\usepackage{amsmath} %用于数学公式(如 \begin{align})
\usepackage{booktabs} %用于表格画线
\usepackage{graphicx} %用于插入图片
\usepackage[top = 1.0in, bottom = 1.0in, left = 1.0in, right = 1.0in]{geometry} %设置页边距

\hypersetup{
	pdfauthor={Orange}
}

\title{新春吃鸡欢乐赛}
\author{Orange}
\date{\today}

\begin{document}
	\heiti
	\maketitle

	\begin{center}
		\setlength{\tabcolsep}{7mm}
		{
			\begin{tabular}{cccc}
				\toprule
					& 地图 & 吃鸡 & 落地成盒 \\
					\hline
					题目类型 & 传统型 & 传统型 & 传统型 \\
					源程序文件名 & compress.cpp & pamffo.cpp & pdd.cpp \\
					输入文件名 & (stdin) & pamffo.in & pdd.in \\
					输出文件名 & (stdout) & pamffo.out & pdd.out \\
					每个测试点时限 & 1 秒 & 1 秒 & 1 秒 \\
					内存限制 & 512 MiB & 512 MiB & 512 MiB \\
					测试点数目 & 20 & 20 & 20 \\
					每个测试点分值 & 5 & 5 & 5 \\
					编译选项 & -O2 & (默认选项) & (默认选项) \\
					比较答案方式 & (默认方式) & (默认方式) & Special Judge \\
				\bottomrule
			\end{tabular}
		}
	\end{center}

	\begin{center}
		\begin{Large}
		请仔细阅读本页内容

		所有测试均\textbf{不}打开 -std=c++11 编译选项

		比较答案的默认方式为全文比较,忽略文末回车,\textbf{但不忽略行末空格}。

		\bigskip
		评测配置:		
		CPU Intel(R) Core(TM) i5-4210H,2.90GHz,
		内存 12 GB
	
		\bigskip
		题目难度与顺序无关

		数据很弱,欢迎水过
		\end{Large}
	\end{center}

	\newpage
	\section {地图 \small{compress.cpp / stdin / stdout}}
	\subsection{题目描述}
	众而周知,$8000\mathrm{m} × 8000\mathrm{m}$ 的地图相当大,
	却还是满足不了萌萌哒选手 compress。

	compress 说:“我要 $x\mathrm{m} × y\mathrm{m}$ 的大地图!”
	既然你无法满足 compress 的地图梦,
	那你就帮他算算他要的地图到底有多大吧!

	\subsection{输入}
	输入包含两行。

	第一行输入一个正整数 $x$,第二行输入一个正整数 $y$。

	\subsection{输出}
	输出包含一行一个正整数 $s$,$s = xy$。

	\subsection{样例}

	\subsubsection{输入}
	8000

	8000

	\subsubsection{输出}
	64000000

	\subsection{数据规模与约定}
	对于 $5\%$ 的数据,$xy$ 在 $\mathrm{int}$ 范围内。

	对于 $10\%$ 的数据,$xy$ 在 $\mathrm{long~long}$ 范围内。

	对于 $20\%$ 的数据,$x, y$ 在 $\mathrm{long~long}$ 范围内。

	对于 $50\%$ 的数据,$x, y \le 10^{5000}$。

	对于 $100\%$ 的数据,$x, y \le 10^{80000}$。

	\subsection{提示}
	stdin / stdout 表示使用标准输入输出。

	评测开启 -O2 编译选项,
	这意味着你可以认为评测机的运行速度比你使用的计算机快不少。

	\newpage
	\section{吃鸡 \small{pamffo.cpp / pamffo.in / pamffo.out}}
	\subsection{题目描述}
	众而周知,每个人都想吃鸡。
	萌萌哒选手 pamffo 决定出来提需求:
	“我们想知道每个人吃了几次鸡!”

	于是这个疯狂的需求就交给你完成:
	你要回答若干次询问。

	\subsection{输入}
	首先输入一行一个整数 $n$。

	接下来有 $n$ 行指令,有两种形式:

	\qquad 指令 1:1 [name] [time]

	\qquad 指令 2:2 [name] [qtime] [time]

	指令 1 表示:在 time 时刻,一位昵称为 name 的人吃鸡了。

	指令 2 表示:在 time 时刻,
	有人询问昵称为 name 的人在 qtime 时刻以及 qtime 之前的吃鸡次数。

	\bigskip
	其中,name 是一个长度不超过 100 的没有空格的非空字符串,
	time 和 qtime 是在 int 范围内的正整数。

	保证 time 递增,保证 qtime $\le$ time。

	\subsection{输出}
	对于每个指令 2,输出一行一个整数,表示吃鸡次数。

	可能会询问从未吃过鸡的人,此时请输出 0。

	\subsection{样例}
	
	\subsubsection{输入}

	6

	2 db 100 100

	1 pamffo 101

	1 db 102

	2 db 100 103

	2 db 102 104

	2 db 101 105

	\subsubsection{输出}

	0

	0

	1

	0

	\subsubsection{解释}
	没什么好解释的。

	\subsection{数据规模与约定}
	对于 $30\%$ 的数据,$n \le 20$。

	对于另外 $30\%$ 的数据,昵称为 short 范围内的正整数。

	对于 $100\%$ 的数据,$n \le 10^5$。

	\newpage
	\section{落地成盒 \small{pdd.cpp / pdd.in / pdd.out}}
	\subsection{题目描述}
	众而周知,落地成盒是件很痛苦的事情,
	然而对于萌萌哒选手 pdd(不是 PDD)来说,
	只要有一种特殊的技巧,就能避免落地成盒。

	这个技巧便是:不停地往前跑,永远不要回头,永远不要往低处走。
	尴尬的是,当不能再往高处前进时,pdd 还是不能避免成盒的命运。
	pdd 希望知道,自己最多能跑到多少地方。

	\bigskip
	整个地图可以抽象为长度为 $n$ 的\textbf{正整数}序列 $\{ a_i \}$,代表每个地方的高度。
	假设 pdd 在位置 $i$ 处,
	那么 pdd 只能跑到 $j$ 处($i < j$,$a_i \le a_j$)。
	由于游戏有随机性,因此 pdd 不仅想知道他最多可能跑到多少地方,
	还想知道从每个地方出发,他最多能跑到多少地方。

	\begin{small}
		一句话题意:求出序列 $\{ a_i \}$ 的最长\textbf{不下降}子序列的长度,
		以及以每个位置开头的\textbf{不下降}子序列的最长长度。
	\end{small}

	\subsection{输入}
	第一行输入一个整数 $n$,表示序列长度。

	接下来一行输入 $n$ 个整数,表示序列中的每个数。 

	\subsection{输出}
	第一行输入一个整数 $n$,表示序列的长度。

	接下来一行输入 $n$ 个整数,表示以每个位置开头的\textbf{不下降}子序列的最长长度。

	\subsection{Special Judge}
	本题使用 Special Judge。

	如果你完成了第一个任务,即第一行输出正确,你将获得 $40\%$ 的分数。

	如果你完成了第二个任务,即第二行输出正确,你将获得 $60\%$ 的分数。

	如果你能完成第二个任务,但不能完成第一个任务,
	请在第一行任意输出一个整数,否则将会错判。

	\subsection{样例}
	
	\subsubsection{输入}
	6

	1 4 2 8 5 7

	\subsubsection{输出}
	4

	4 3 3 1 2 1

	\subsubsection{解释}
	从 $a_1 = 1$ 出发,最长不下降子序列为 1 4 5 7 或者 1 2 5 7。
	
	从 $a_2 = 4$ 出发,最长不下降子序列为 4 5 7。
	
	从 $a_3 = 2$ 出发,最长不下降子序列为 2 5 7。

	从 $a_4 = 8$ 出发,最长不下降子序列为 8。

	从 $a_5 = 5$ 出发,最长不下降子序列为 5 7。

	从 $a_6 = 7$ 出发,最长不下降子序列为 7。

	\subsection{数据规模与约定}
	对于 $20\%$ 的数据,$n \le 20$。

	对于 $50\%$ 的数据,$n \le 5 × 10^3$。

	对于 $100\%$ 的数据,$n \le 10^6$,$a_i \in \mathrm{int}$。

	\subsection{提示}
	本题使用 Special Judge,将会忽略行末空格及文末回车。

	请注意常数因子带来的程序效率上的影响。
\end{document}