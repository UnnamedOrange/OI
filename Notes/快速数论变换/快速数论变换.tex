\documentclass[12pt, UTF8]{article}
\usepackage{ctex}
\usepackage{hyperref} %用于设置 PDF 的信息
\usepackage{setspace} %用于设置行间距
\usepackage{listings} %用于代码高亮
\usepackage{xcolor} %用于处理颜色
\usepackage{ulem} %用于各种线
\usepackage{amsmath} %用于数学公式(如 \begin{align})
\usepackage[top = 1.0in, bottom = 1.0in, left = 1.0in, right = 1.0in]{geometry} %设置页边距

\hypersetup{hidelinks}
\hypersetup{
    pdfauthor={Orange}
}

\setmonofont{Consolas} %设置字体
\lstset{
    basicstyle = \footnotesize\ttfamily,
    numbers = left,
    numberstyle = \tiny,
    keywordstyle = \color{blue!100},
    commentstyle = \color{red!50!green!50!blue!50},
    rulesepcolor = \color{red!20!green!20!blue!20},
    xleftmargin = 2em, xrightmargin = 2em, aboveskip = 1em,
}

\title{快速数论变换 FNTT}
\author{Orange}
\date{\today}

\begin{document}
    \heiti
    \maketitle

    \section{摘要}
    离散傅里叶变换(DFT)是解决卷积问题的有力途径,
    但是它涉及到了复数和三角函数等浮点数运算,
    这必然会产生浮点数运算常见的问题:
    计算速度慢,浮点误差大。

    回忆使用 DFT 解决多项式乘法的过程,
    我们实际上是利用 $n$ 次单位复数根的特殊性质
    使得整个过程可分治,
    最终得到了快速傅里叶变换(FFT)算法。
    这个特殊性质到底是什么呢?
    有没有代替品?这都是我们要弄清楚的问题。

    本文从 FFT 的推导过程出发,
    在模意义下的整数域中讨论快速数论变换,并给出有效的实现。

    \bigskip
    \leftline{关键词:数论变换(Number Theoretic Transform,NTT)}

    \section{快速复习}
    \subsection{离散傅里叶变换}
    简单地说,离散傅里叶变换是
    将在时域上长度有限且离散的信息转换到频域上,
    得到的频域信息仍然是离散的。

    选择 DFT 来解决卷积运算问题的主要理由是
    DFT 满足时域卷积定理和频域卷积定理。
    这样,实质为卷积的多项式乘法便能通过 DFT 将问题转换成
    时间复杂度为 $O(n)$ 的点积运算。

    DFT 及离散傅里叶逆变换(IDFT)是整个算法的时间瓶颈,
    因为点积运算的时间复杂度为 $O(n)$,
    而朴素的 DFT 的时间复杂度为 $O(n^2)$;
    对应的,通过逆矩阵的方法,也可以得到朴素的 IDFT 算法,
    时间复杂度为 $O(n^2)$。

    \bigskip
    DFT 以及 IDFT 的公式为:
    \begin{align}
        X_k = \sum_{j = 0}^{n - 1} x_j · w_n^{jk} &
        \qquad (0 \le k < n) \notag
        \\
        x_k = \frac {1} {n} \sum_{j = 0}^{n - 1}X_j · w_n^{-jk} &
        \qquad (0 \le k < n) \notag
    \end{align}
    其中 $x$ 代表时域序列,$X$ 代表频域序列。

    \bigskip
    利用 $n$ 次单位复数根的特殊性质,
    可以将 DFT 的时间复杂度优化为 $O(n \log n)$,
    IDFT 情况类似。
    就这样,我们得到了一个时间复杂度为 $O(n \log n)$ 的多项式乘法算法。
    \footnote{具体的内容可以参看离散傅里叶变换。}

    \subsection{模 $m$ 意义下的 $n$ 次单位根}
    \subsubsection{定义}
    在模 $m$ 意义下,对于非零元素 $a$,
    如果存在正整数 $n$,使得 $a^n \equiv 1 \pmod {m}$,
    则称 $a$ 是模 $m$ 意义下的 $n$ 次单位根。

    \subsubsection{阶}
    对于两个互质的数 $a$,$m$,
    称最小的满足 $a^x \equiv 1 \pmod {m}$ 的 $x$ 为 $a$ 在模 $m$ 意义下的阶,
    记作 $x = \delta_m(a)$ \footnote{$\delta$ 读作 delta。};
    而称 $a$ 是对模 $m$ 的 $n$ 阶本原单位根。

    \subsubsection{原根}
    由欧拉定理,不难得到 $a^{\varphi(m)} \equiv 1 \pmod {m}$,
    但是 $\varphi(m)$ 是 $a$ 在模 $m$ 意义下的阶吗?不一定。

    \bigskip
    如果对于 $x$,满足 $x^{\varphi(m)} \equiv 1 \pmod {m}$,
    那么我们称 $x$ 为 $m$ 的一个原根。

    \subsection{多项式}
    \subsubsection{次数界}
    定义一个多项式的次数界为其最高项的次数加一,记为 $N_A$。

    \subsubsection{点值运算与插值运算}
    点值运算指对一个次数界为 $N_A$ 的多项式 $A(x)$
    代入至少 $N_A$ 个不同参数 $x$,
    得到至少 $N_A$ 个形如 $(x_j, A(x_j))$ 的点值对。
    这个操作得到的结果称为多项式的点值表示。

    插值运算指通过至少 $N_A$ 个点值对
    计算出一个次数界为 $N_A$ 的多项式的各项系数。
    当点值对不足 $N_A$ 个时,对应的多项式将会不唯一。

    \bigskip
    DFT 实质上就是一个点值运算;IDFT 实质上就是一个插值运算。

    \subsubsection{系数表示的多项式乘法与点值表示的多项式乘法}
    对两个使用系数表示的多项式做乘法,
    实际上是做一次两个向量的卷积,
    朴素计算的时间复杂度为 $O(n^2)$。

    对两个使用点值表示的多项式做乘法,
    首先要保证点值对个数相同(设为 $(x, y)$),
    其次要保证各点值对的 $x$ 相同,
    最后只需要将各个 $y$ 对应相乘即可。时间复杂度为 $O(n)$。

    \section{数论变换}
    \subsection{代数系统}
    为什么我们要选择运算量巨大的复数域?
    因为我们要进行 DFT,要使用 $n$ 次单位复数根的几个特殊性质:
    \footnote{具体的内容可以参看快速傅里叶变换的相关文献。}
    \begin{gather}
        w_n^0 = w_n^n = 1
        \\
        (w_n^k)^2 = w_{\frac {n} {2}}^k
        \\
        \sum_{j = 0}^{n - 1} w_n^{jk} = 0 \qquad (k \ne 0)
    \end{gather}

    $(1)$ 代表的性质使得 DFT 满足\textbf{循环卷积特性}。

    $(2)$ 代表\textbf{消去引理}。
    结合 $(1)$ 和 $(2)$,我们可以得到\textbf{折半引理}。

    $(3)$ 代表\textbf{求和引理},可以由它推导出 DFT 的逆运算 IDFT。

    \bigskip
    已经证明,在复数域中,DFT 是唯一满足循环卷积特性的变换。

    \bigskip
    能否在别的代数系统中找到类似的变换呢?

    \subsection{数论变换}
    定义长度为 $n$ 的序列 $x$,有以下变换:
    \begin{equation*}
        X_k = \sum_{j = 0}^{n - 1} x_j a^{jk} \pmod {m}
        \qquad (0 \le k < n)
    \end{equation*}

    我们称上面的变换为\textbf{数论变换(Number-Theoretic Transform,NTT)}。

    \bigskip
    应当注意到,正如数论变换的名字,
    它涉及到的所有数都是在模 $m$ 意义下的整数。

    \subsubsection{数论逆变换}
    仿照 DFT 转化为代表线性组方程的矩阵乘法的过程,
    我们对 NTT 也进行相同的转化:
    \begin{equation*}
        \begin{bmatrix}
            1 & 1 & 1 & \cdots & 1
            \\
            1 & a & a^2 & \cdots & a^{n - 1}
            \\
            1 & a^2 & a^4 & \cdots & a^{2(n - 1)}
            \\
            \vdots & \vdots & \vdots & \ddots & \vdots
            \\
            1 & a^{n - 1} & a^{2(n - 1)} & \cdots & a^{(n - 1)^2}
        \end{bmatrix}
        \begin{bmatrix}
            x_0
            \\
            x_1
            \\
            x_2
            \\
            \vdots
            \\
            x_{n - 1}
        \end{bmatrix}
        \equiv
        \begin{bmatrix}
            X_0
            \\
            X_1
            \\
            X_2
            \\
            \vdots
            \\
            X_{n - 1}
        \end{bmatrix}
        \pmod {m}
    \end{equation*}

    记作:
    \begin{equation*}
        V_n x \equiv X \pmod {m}
    \end{equation*}

    \bigskip
    我们要找到满足以下条件的 $V_n^{-1}$:
    \begin{equation*}
        V_n V_n^{-1} \equiv I_n \pmod {m}
    \end{equation*}
    其中 $I_n$ 为 $n$ 阶单位矩阵。

    \bigskip
    仿照 DFT 构造逆矩阵,我们能否构造出下面这个矩阵作为 $V_n^{-1}$?
    \begin{equation*}
        \begin{bmatrix}
            n^{-1} & n^{-1} & n^{-1} & \cdots & n^{-1}
            \\
            n^{-1} & a^{-1}n^{-1} & a^{-2}n^{-1} & \cdots & a^{-(n - 1)}n^{-1}
            \\
            n^{-1} & a^{-2}n^{-1} & a^{-4}n^{-1} & \cdots & a^{-2(n - 1)}n^{-1}
            \\
            \vdots & \vdots & \vdots & \ddots & \vdots
            \\
            n^{-1} & a^{-(n - 1)}n^{-1} & a^{-2(n - 1)}n^{-1} &
            \cdots & a^{-(n - 1)^2}n^{-1}
        \end{bmatrix}
    \end{equation*}
    其中 $n^{-1}$ 表示 $n$ 的逆元,
    $a^{-k}$ 表示 $a$ 的逆元的 $k$ 次方。

    我们只在这里证明 $V_n^{-1}$ 为其逆矩阵的必要性。

    \bigskip
    光是 $n^{-1}$ 就已经有难度了:对于模数 $m$,
    如果 $m$ 不是质数,那么 $n$ 和 $a$ 不一定存在逆元。
    到现在,我们并没有对 $m$ 和 $a$ 作出任何限制,
    所以我们可以任选一个 $m$。为了保证存在逆元,
    \textbf{我们强行规定 $m$ 为一个质数,
    并且重新定义符号 $p$ 代表新的模数。}

    \bigskip
    现在只需要证明:
    \begin{equation*}
        \sum_{k = 0}^{n - 1} V_{n ~ (i, k)} · V^{-1}_{n ~ (k, j)}
        \equiv
        [i = j]
        \pmod {p}
    \end{equation*}

    即:
    \begin{equation*}
        \sum_{k = 0}^{n - 1} a^{ki} · a^{-kj} · n^{-1}
        \equiv
        [i = j]
        \pmod {p}
    \end{equation*}

    左式化简得:
    \begin{equation*}
        n^{-1} · \sum_{k = 0}^{n - 1} a^{k(i - j)}
    \end{equation*}

    当 $i = j$ 时,上式显然同余 $1$。
    当 $i \ne j$ 时,如果我们有:
    \begin{equation*}
        \sum_{k = 0}^{n - 1} a^{k(i - j)} \equiv 0 \pmod {p}
    \end{equation*}
    即可得证。

    \bigskip
    可以将左式看作等比数列求和,化简左式,易得:
    \begin{equation*}
        \frac {1 - a^{n(i - j)}} {1 - a^{(i - j)}}
    \end{equation*}

    现在的问题转化为了求证:
    \begin{equation*}
        a^n \equiv 1 \pmod {p}
    \end{equation*}

    要使上式成立,必须有:
    \begin{equation*}
        a^k \not \equiv 1 \pmod {p} \qquad (1 \le k < n)
    \end{equation*}
    否则会有 $a^n \equiv 0 \pmod {p}$。

    注意,$a$ 的取值也是任选的。
    要使 $a^n \equiv 1 \pmod {p}$ 成立,
    等价于 \textbf{$a$ 是模 $p$ 意义下的 $n$ 阶本原单位根。}
    \textbf{我们强行规定 $a$ 为模 $p$ 意义下的 $n$ 阶本原单位根,
    并且重新定义符号 $g$ 代表新的底数},
    则原命题成立。

    \subsubsection{$p$ 和 $g$ 的取值}
    先明确目前我们对 $p$ 和 $g$ 的要求。目前,
    \textbf{$p$ 是一个质数,而 $g$ 是 $p$ 的 $n$ 阶本原单位根。}

    \bigskip
    我们有:
    \begin{equation*}
        g^n \equiv 1 \pmod {p}
    \end{equation*}

    由 $n$ 阶本原单位根的定义和二次探测定理,有:
    \begin{equation*}
        g^{\frac {n} {2}} \equiv -1 \pmod {p}
    \end{equation*}

    如果给定一个 $n$,
    事实上直接求出 $p$ 的 $n$ 阶本原单位根并不是一个容易的事,
    但我们可以考虑使用原根:
    \begin{gather*}
        a^{p - 1} \equiv 1 \pmod {p}
        \\
        a^i \not \equiv a^j \pmod {p} \qquad (0 \le i < j < p - 1)
    \end{gather*}

    如果我们有:
    \begin{equation*}
        p - 1 = k · n
    \end{equation*}

    令 $g_n \equiv a^k \pmod {p}$,则 $g_n$ 就是 $p$ 的一个 $n$ 阶本原单位根。
    \textbf{所以我们可以先计算出 $p$ 的一个原根 $g$,
    再令 $g_n = g^{\frac {p - 1} {n}}$ 作为底数。}

    \bigskip
    注意到,$n$ 可能是一定数据范围内的任意取值,
    所以要使 $p - 1 = k · n$ 对于任意的 $n$ 都满足是不可能的。

    对比快速傅里叶变换,我们要求 $n = 2^t$ 是为了让问题可以不断分治。
    这里我们也需要用到这个思路。
    \textbf{我们强制规定 $n = 2^t$,如果原长度不足 $2^t$,则补 $0$ 至 $n$ 位。}

    现在 $n = 2^t$,如果 $p - 1 = k · 2^t$,问题就变得很容易了。
    \textbf{所以我们强制规定 $p = k · 2^t + 1$,
    这能够解决所有 $n \le 2^t$ 规模的问题。}

    \subsubsection{小结数论逆变换}
    明确目前 $p$ 和 $g$ 的要求。
    目前,
    $p = k · 2^t + 1$ 且为质数,
    $g = k · 2^{t'}$,
    能够解决 $n = \frac {2^t} {2^{t'}}$ 问题。

    再次明确为什么要这么规定 $p$ 和 $g$ 的取值:
    令 $p$ 为质数的原因是保证存在 $n^{-1}$
    且保证存在原根。
    令 $g$ 为这个值的原因
    是保证 $g$ 是 $p$ 的一个 $n$ 阶本原单位根,
    使得 NTT 的逆运算 INTT 存在。

    \bigskip
    称下列变换为数论变换和数论逆变换
    (Inverse Number Theoretic Transform,INTT):
    \begin{gather*}
        X_k = \sum_{i = 0}^{n - 1} x_i · g^{ik} \pmod {p}
        \\
        x_k = n^{-1} \sum_{i = 0}^{n - 1} X_i · g^{-ik} \pmod {p}
    \end{gather*}
    其中 $p = k · 2^t + 1$ 且为质数,$n = 2^{t'}$,$g = k · 2^{t - t'}$。

    \subsection{快速数论变换}
    容易得到,直接进行 NTT 和 INTT 的时间复杂度为 $O(n^2)$。
    我们可以用类似快速傅里叶变换的方法,
    进行快速数论变换(Fast Number Theoretic Transform,FNTT)。

    \subsubsection{循环卷积特性和三个引理}
    在快速傅里叶变换中,我们用到了 $n$ 次单位复数根带来的循环卷积特性
    和三个引理:

    \qquad {1.消去引理}

    \qquad {2.折半引理}

    \qquad {3.求和引理}

    \bigskip
    我们实际上已经证明了求和引理。
    不难证明,$n$ 阶本原单位根也会给 NTT 带来循环卷积特性:
    \begin{equation*}
        g_n^0 \equiv g_n^n \equiv 1 \pmod {p}
    \end{equation*}
    其中 $g_n$ 为 $n$ 对应的 $n$ 阶本原单位根。

    \bigskip
    下面让我们来证明消去引理和折半引理。
    ($p = k · 2^t + 1$,$n = 2^{t - 1}$,$g_n = k · 2^{t - t'}$,
    设 $g$ 为 $p$ 的一个原根)

    对于 $g_n^d ~ (d \bmod 2 = 0)$,有:
    \begin{equation*}
        g_n^d \equiv g^{d (k · 2^{t - t'})}
        \equiv g^{\frac {d} {2} (k · 2^{t - t' + 1})}
        \equiv g_{\frac {n} {2}}^{\frac {d} {2}}
        \pmod {p}
    \end{equation*}
    即证明了消去引理。

    \bigskip
    我们有:
    \begin{equation*}
        g_n^k \equiv g_n^{k + n} \pmod {p}
    \end{equation*}

    对于 $g_{\frac {n} {2}}^{k} ~ (0 \le k < n)$,
    由上式,易得在模 $p$ 意义下只有 $\frac {n} {2}$ 个不同的数。
    即证明了折半引理。

    \subsubsection{快速数论变换}
    能够实现快速傅里叶变换,得益于循环卷积特性和三个引理。
    现在在模 $p$ 意义下,循环卷积特性和三个引理也得证了。
    能否用相同的方法推导出快速数论变换呢?
    答案是肯定的。

    \bigskip
    下面我们用 $=$ 代替 $\equiv$,并且省略模数。
    设原多项式:
    \begin{equation*}
        X_k = x_0 + x_1 · g_n^{k} + x_2 · g_n^{2k} +
        \cdots + x_{n - 1} · g_n^{(n - 1)k}
    \end{equation*}

    可以分解成两个多项式的线性变换:
    \begin{gather*}
        X_k^{[0]} = x_0 + x_2 · g_n^{2k} + \cdots
        + x_{n - 2} · g_n^{(n - 2)k}
        \\
        X_k^{[1]} = x_1 + x_3 · g_n^{2k} + \cdots
        + x_{n - 1} · g_n^{(n - 2)k}
        \\
        X_k = X_k^{[0]} + g_n^k · X_k^{[1]}
    \end{gather*}

    由消去引理,两个多项式分别为:
    \begin{gather*}
        X_k^{[0]} = x_0 + x_2 · g_{\frac {n}{2}}^{k} + \cdots
        + x_{n - 2} · g_{\frac {n}{2}}^{\frac {n - 2} {2} k}
        \\
        X_k^{[1]} = x_1 + x_3 · g_{\frac {n}{2}}^{k} + \cdots
        + x_{n - 1} · g_{\frac {n}{2}}^{\frac {n - 2} {2} k}
    \end{gather*}

    当 $k$ 取值 $0 \le k < n$ 时,可以发现,有:
    \begin{gather*}
        \begin{matrix}
            X_{k'}^{[0]} = X_{k' + \frac {n} {2}}^{[0]}
            \\
            X_{k'}^{[1]} = X_{k' + \frac {n} {2}}^{[1]}
        \end{matrix}
        \qquad
        (0 \le k' < \frac {n} {2})
    \end{gather*}

    代入原多项式,可以发现:
    \begin{gather*}
        \begin{matrix}
            X_{k'} = X_{k'}^{[0]} + g_n^{k'} · X_{k'}^{[1]}
            \\
            X_{k' + \frac {n} {2}}
            = X_{k' + \frac {n} {2}}^{[0]} +
            g_n^{k' + \frac {n} {2}} · X_{k' + \frac {n} {2}}^{[1]}
            = X_{k'}^{[0]} - g_n^{k'} · X_{k'}^{[1]}
        \end{matrix}
        \qquad
        (0 \le k' < \frac {n} {2})
    \end{gather*}

    至此,可由主定理得时间复杂度为 $O(n \log n)$。

    \bigskip
    相应的,INTT 也使用类似的方式进行转化,
    具体内容不在阐述。

    \subsubsection{实现}
    由于我们已经知道了 FFT 的实现,所以完全可以照搬 FFT 的过程:
    \textbf{FNTT 与 FFT 在实现上差距并不大。}
    这里直接给出代码:

    \lstset{language=C++}
    \begin{lstlisting}
struct FNTT
{
    static const int mod = 998244353;
    static const int root = 3;
    static const int base = 119;
    static const int limit = 23;
    static inline int power(int x, int y)
    {
        int ret = 1;
        while (y)
        {
            if (y & 1) ret = (long long)ret * x % mod;
            x = (long long)x * x % mod;
            y >>= 1;
        }
        return ret;
    }
    int n, logn;
    inline int revbit(int x)
    {
        int ret = 0;
        for (int i = 0; i < logn; i++)
            ret = (ret << 1) | (bool)(x & (1 << i));
        return ret;
    }
    FNTT(int* a, int logn, int sig) : n(1 << logn), logn(logn)
    {
        for (int i = 0; i < n; i++)
        {
            int t = revbit(i);
            if (i < t) std::swap(a[i], a[t]);
        }
        for (int i = 1; i <= logn; i++)
        {
            int S = 1 << i;
            int half = S >> 1;
            int g1 = power(root, base * (1 << (23 - i)));
            if (sig == -1) g1 = power(g1, mod - 2);
            for (int j = 0; j < n; j += S)
            {
                int* A = a + j;
                int g = 1;
                for (int k = 0; k < half; k++)
                {
                    int t = (long long)A[k + half] * g % mod;
                    A[k + half] = ((A[k] - t) % mod + mod) % mod;
                    A[k] = (A[k] + t) % mod;
                    g = (long long)g * g1 % mod;
                }
            }
        }
    }
};
    \end{lstlisting}

    \section{小结}
    NTT 是在模意义下进行的运算,它的主要作用还是计算多项式乘法。
    若多项式乘法的结果保证为小于模数的非负整数,
    则可以使用 NTT 代替 FFT;
    如果题目要求对一个适用于 NTT 的模数取模,则只能使用 NTT。

    有时候,受代码常数影响,NTT 比 FFT 运行得要慢一点,
    但是整体上 NTT 运行得要快一点,且内存占用比 FFT 少得多。

    NTT 不受浮点误差影响。

    \newpage
    \section{附表:部分质数与其对应的原根}
    \begin{equation*}
        p = k · 2^t + 1
    \end{equation*}

    $k$ 和 $g$\footnote{$g$ 为 $p$ 的原根。} 均对实现没有影响。
    $t$ 决定了能够解决的问题的最大规模。

    \bigskip
    \begin{center}
        \setlength{\tabcolsep}{10mm}
        \begin{tabular}{|c|c|c|c|}
        \hline
        p & k & t & g\\
        \hline
        998244353 & 119 & 23 & 3\\
        \hline
        1004535809 & 479 & 21 & 3\\
        \hline
        \end{tabular}
    \end{center}

    如果题目给定的模数不是一个满足要求的质数,
    就需要做这么一个思维转化:
    既然题目要求求模,那就可以保证所有的数小于 $m$,
    两个数的乘积一定小于 $(m - 1)^2$,
    而卷积的结果就一定小于 $n(m - 1)^2$。
    如果 NTT 的模数是一个符合条件的大于 $n(m - 1)^2$ 的质数,
    问题也就迎刃而解了。

    但是可能很难找到这样的质数,找到了在计算机中也存不下。
    所以需要找 $b$ 个符合条件的质数,使得它们的乘积大于等于 $n(m - 1)^2$,
    最后使用中国剩余定理合并。

    \section{参考文献}
    \thanks {Miskcoo,从多项式乘法到快速傅里叶变换,2015.}

\end{document}