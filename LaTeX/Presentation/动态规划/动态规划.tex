\documentclass[UTF8]{beamer}
\usepackage{ctex}
\usepackage{hyperref} %用于设置 PDF 的信息
\usepackage{setspace} %用于设置行间距
\usepackage{listings} %用于代码高亮 
\usepackage{xcolor} %用于处理颜色
\usepackage{ulem} %用于各种线
\usepackage{amsmath} %用于数学公式(如 \begin{align})
\usepackage{booktabs} %用于表格画线
\usepackage{graphicx} %用于插入图片
%\usepackage[top = 1.0in, bottom = 1.0in, left = 1.0in, right = 1.0in]{geometry} %设置页边距

\hypersetup{
  pdfauthor={Orange}
}

\usetheme[secheader]{Madrid}

\title{动态规划}
\author{Orange}
\institute{CQ No.11 High}
\date{2018 年 1 月 28 日}

\begin{document}
	\heiti %设置字体
	\setlength{\parindent}{2em} %中文缩进两个汉字位

	\begin{frame} %标题页
		\titlepage
	\end{frame}
	
	\subsubsection{写在前面}
	\begin{frame}
		\frametitle{\insertsubsubsection} %标题
		虽然说是讲动态规划,但是可能会涉及到很多其它东西,甚至讲得比动态规划还深。
		所以你们看着办好啦。
		
		大家一定要踊跃一点。
		\\\pause %停一下
		\vspace{5mm} %空一点
		另外不要看《信息学奥赛一本通》,谢谢,不然你的动态规划只会学得一塌糊涂。
		\\\pause
		\vspace{5mm}
		希望大家在最后能够感慨:啊动态规划就是这么个简单东西。
	\end{frame}

	%\begin{frame}
	%	\frametitle{\insertshorttitle} %将标题页主标题设为当前的标题
	%	\tableofcontents %目录
	%\end{frame}

	\section{入门}
	\subsection{数字三角形问题}

	\begin{frame}
		\frametitle{\insertsubsection} %将小节标题设为当前的标题
		有一个由非负整数组成的三角形,形状与杨辉三角类似:
		第一行只有一个数,除了最下面的一行,每个数的左下方和右下方各有一个数。
		\\
		从第一行的数开始,每次可以往左下或右下走一格,直到走到最下行。
		把沿途经过的数全部加起来,如何走才能使得这个和尽量大?
		\\
		\begin{center} %设置居中内容
			{\LARGE $n \le 1000$} %加大字号,命令有大小写区别
		\end{center}
	\end{frame}

	\subsubsection{方法〇~——~一个很好写的方法} %小空格
	\begin{frame}
		\frametitle{\insertsubsubsection}
		对任意一个位置,哪一边的数字大,我们就走哪边。
		\\\pause
		然而很明显这是错的。
		\\\pause
		我们把这种方法叫做 \textbf{乱贪心}。 %加粗
		对于任何一个贪心的方法,在时间等条件允许时都要仔细证明。
	\end{frame}

	\subsubsection{方法①}
	\begin{frame}
		\frametitle{\insertsubsubsection}
		大家都会搜索吧?毫无疑问,搜索出来的答案一定是对的。
		但是时间复杂度如何呢?
		\\\pause
		\vspace{5mm}
		总共有~$2^{n - 1}$~条路径。
		\begin{center}
			$O(2^{n})$
		\end{center}
	\end{frame}

	\subsubsection{方法②}
	\begin{frame}
		\frametitle{\insertsubsubsection}
		当然是动态规划啦!
		\\
		从这个问题来看,\textbf{我们使用动态规划的理由之一是解决某些爆搜难以解决的问题。}
		\\
		所以我们应当思考,\textbf{为什么爆搜这么慢}。
	\end{frame}

	\begin{frame}
		\frametitle{\insertsubsubsection}
		我们来想想搜索是怎么搜的。
		\\
		常见方法是记录从上往下的和,走到底时更新一下答案。
		但我们试试 \textbf{反着思考,从下往上走}。
		\\
		假设我们正在第~$i$~行第~$j$~列。我们想的是,左边或者右边,
		\textbf{哪边走出来的路径的和最大},我们就走哪边,或者是,从哪边来。
		因为我们在递归求解,\textbf{所以我们每调用一次函数,就得到一次子问题的答案。}
		\\\pause
		然后我们就得到了第~$i$~行第~$j$~列的答案,
		于是我们就高高兴兴地回到了第~$i - 1$~行。
	\end{frame}

	\begin{frame}
		\frametitle{\insertsubsubsection}
		然后你就惊讶地发现:你会不止一次来到第~$i$~行第~$j$~列,
		因为从第~$1$~行第~$1$~列到它的路径不是唯一的。
		\\
		然后你就又进行了许多次相同的运算……
	\end{frame}

	\begin{frame}
		\frametitle{\insertsubsubsection}
		我们为什么不保存这个答案呢?
		\\
		\textbf{设~$f_{i, j}$~表示第~$i$~行第~$j$~列的答案。}
		如果我们已经运算出来了,我们就直接用保存的值。
		如果没有,那只好老老实实地来算了……
		\\\pause
		看上去很诱人,但是效率如何呢?
	\end{frame}

	\begin{frame}
		\frametitle{\insertsubsubsection}
		当一个位置已经计算出来后,我们不会再往下面走;
		而计算底层位置的答案只需要常数时间。
		\\
		可以看作每次用最底层去更新倒数第二层,然后删去最底层。
		每次操作的时间复杂度为~$O(n)$,总共有~$n$~层,
		所以时间复杂度为~$O(n^2)$。
	\end{frame}

	\begin{frame}
		\frametitle{\insertsubsubsection}
		以上方法的核心思想是:
		\textbf{把已经算过的东西保存下来,不要重复计算;
		用已经算过的东西去计算没有算过的东西。}
		\\\pause
		这就是\textbf{动态规划的通俗描述}。
	\end{frame}

	\subsubsection{方法②}
	\begin{frame}
		\frametitle{\insertsubsubsection}
		于是我们就有了这么个写法:在递归函数的开头检查是否已经算过。
		\\\pause
		由于这种写法很像搜索,我们称这种写法为\textbf{记忆化搜索},简称记搜。
	\end{frame}

	\begin{frame}
		\frametitle{\insertsubsubsection}
		不难发现,我们可以直接从下往上推算,用两层循环就能搞定。
		\\
		我们称这种写法为\textbf{递推法}。
		\\\pause
		递推时,我们从倒数第二层开始推算,用下一层的答案来更新当前层。
		这种用别人的答案来更新自己的答案的做法叫做\textbf{填表法}。
		有一个好记的名字,叫做“人人为我”。
		\\\pause
		不难发现,这种方法的时间复杂度和记忆化搜索一致。
		之前我们计算时间复杂度时其实就是用的这种方法。
		一般而言,\textbf{递推法比记忆化搜索要快},因为它避免了递归的时间开销。
		\\\pause
		但有时,有的问题难以用递推法解决;有的问题受题目性质影响,记忆化搜索不会遍历到所有状态,
		这种问题使用记搜反而快。
	\end{frame}

	\subsubsection{技巧与注意事项}
	\begin{frame}
		\frametitle{\insertsubsubsection}
		memset~函数的使用?
		\\\pause
		动态规划与“递推”的关系?
		\\\pause
		是不是动态规划都像解决这个问题一样能够大幅降低时间复杂度?
		\\\pause
		做很有可能是动态规划的题目的正确思考顺序?
	\end{frame}

	\subsubsection{动态规划的一些基本概念}
	\begin{frame}
		\frametitle{\insertsubsubsection}
		\Large {对于任意一个非底层位置, %状态
		\\
		\vspace{5mm}
		我们有两个选择:从左边来,或者从右边来, %决策
		\\
		\vspace{6mm}
		选择答案最大的路径来更新当前位置的答案。} %状态转移
	\end{frame}

	\subsection{状态设计}
	\subsubsection{数字三角形问题~2}
	\begin{frame}
		\frametitle{\insertsubsubsection}
		有一个由非负整数组成的三角形,形状与杨辉三角类似:
		第一行只有一个数,除了最下面的一行,每个数的左下方和右下方各有一个数。
		\\
		从第一行的数开始,每次可以往左下或右下走一格,直到走到最下行。
		把沿途经过的数全部加起来,如何走才能使得这个和\underline{的个位}尽量大?
		\\
		\begin{center}
			{\LARGE $n \le 1000$}
		\end{center}
	\end{frame}

	\begin{frame}
		\frametitle{\insertsubsubsection}
		继续套用第一个问题的状态设计,发现会有一些问题。
		\\\pause
		两个个位数最大的数加起来得到的数的个位不一定大。
		\\\pause
		我们称这种情况为不满足\textbf{最优子结构性质}。
		\\
		\textbf{动态规划的状态转移必须满足最优子结构性质。}
	\end{frame}

	\begin{frame}
		\frametitle{\insertsubsubsection}
		设~$f_{i, j, k}$~表示第~$i$~行第~$j$~列的数个位是否可能为~$k$。
		\\\pause
		转移显然,但是答案是什么?最底下的那一行的~$f$~又应是如何的?
		\\\pause
		\vspace{5mm}
		\textbf{这道题给我们的启发:动态规划必须做好三件事:
		\\
		\qquad ①明确状态
		\\
		\qquad ②知道边界的状态
		\\
		\qquad ③能用状态算出我们需要的答案}
	\end{frame}

	\begin{frame}
		\frametitle{\insertsubsubsection}
		这个算法的时间复杂度是多少?
		\\\pause
		\begin{center}
			$O(100 n^2)$
		\end{center}
		\ %解决 center 后换行的问题
		\\\pause
		一般来说,动态规划的时间复杂度为:
		\\
		\begin{center}
			\textbf{状态总数~×~决策个数~×~决策时间}。
		\end{center}
	\end{frame}

	\subsection{路径输出}
	\subsubsection{数字三角形问题~1~Ex}
	\begin{frame}
		\frametitle{\insertsubsubsection}
		如果要求输出路径,该怎么做呢?
		\\\pause
		用~$pre_{i, j}$~记录~$f_{i, j}$~由哪个状态转移而来。
		\textbf{因为我们的决策是选择左边或者选择右边,所以从哪儿转移而来就肯定选了哪个的。}
		\\\pause
		如果要求输出从下到上的路径,只能用栈保存路径。
		当然这也意味着你可以递归输出,因为递归的数据结构基础就是栈。
		\\\pause
		绝大多数要输出一种具体方案的动态规划都是这么干的。
	\end{frame}

	\section{图论基础(初步)}
	\begin{frame}
		所以最基础的动态规划就这样啦!是不是很简单?
		\\
		要进行接下来的学习,得先了解一点点图论的内容。
	\end{frame}

	\subsection{概念}
	\subsubsection{概念}
	\begin{frame}
		\frametitle{\insertsubsubsection}
		图的逻辑概念为~$G = (V, E)$。
		\\
		指的是图~$G$~由(顶)点集合~$V$~和边的集合~$E$~组成。
		\\\pause
		\vspace{5mm}
		说大白话,让我给你们画一下就知道了。
	\end{frame}

	\subsubsection{分类}
	\begin{frame}
		\frametitle{\insertsubsubsection}
		图大概可以分为两类:
		\\
		\qquad ·~有向图
		\\
		\qquad ·~无向图
		\\\pause
		\vspace{5mm}
		其中有一种特殊的图,叫做有向无环图(Directed Acyclic Graph, DAG)。
	\end{frame}

	\subsection{有向无环图}
	\subsubsection{无后效性}
	\begin{frame}
		\frametitle{\insertsubsubsection}
		什么叫做无后效性呢?让我们回到数字三角形问题~1。
		\begin{center}
			\Large {左边或者右边 \underline{选一条答案最大的}。}
		\end{center}
		\ \\
		如果某个状态的答案一旦计算出来就不会再改变,就称这个状态是\textbf{无后效性}的。
		\\\pause
		DAG~是一种典型的无后效性的图,换句话说,\textbf{从~DAG~上的任一点出发都不能回到出发点}。
		\\\pause
		而 \textbf{动态规划要求满足两个性质:最优子结构性质和无后效性},所以~DAG~和动态规划是息息相关的。
	\end{frame}

	\subsection{拓扑序}
	\subsubsection{拓扑序}
	\begin{frame}
		%\frametitle{\insertsubsubsection}
		我太懒了,所以剩下的都没有做了……
		\\
		凑合看看上课的笔记吧。
	\end{frame}
\end{document}