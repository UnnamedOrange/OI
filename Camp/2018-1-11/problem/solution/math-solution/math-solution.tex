% Created 2017-01-18 Wed 17:02

\documentclass[12pt,a4paper]{article}
\usepackage[top = 1.0in, bottom = 1.0in, left = 1.0in, right = 1.0in]{geometry}

\usepackage{hyperref}
\hypersetup{
  colorlinks=true,
  linkcolor=[rgb]{0,0.37,0.53},
  citecolor=[rgb]{0,0.47,0.68},
  filecolor=[rgb]{0,0.37,0.53},
  urlcolor=[rgb]{0,0.37,0.53},
  pagebackref=true,
  linktoc=all,}

\usepackage{hyperref}
\usepackage[utf8]{inputenc}
\usepackage{ctex}
\usepackage{fixltx2e}
\usepackage{graphicx}
\usepackage{longtable}
\usepackage{float}
\usepackage{wrapfig}
\usepackage{rotating}
\usepackage[normalem]{ulem}
\usepackage{amsmath}
\usepackage{textcomp}
\usepackage{marvosym}
\usepackage{wasysym}
\usepackage{multicol}
\usepackage{amssymb}
\tolerance=1000
\date{\today}
\author{fuboat}
\title{Math Solution}
\hypersetup{
  pdfkeywords={},
  pdfsubject={},
  pdfcreator={Emacs 25.1.1 (Org mode 8.2.10)}}
\begin{document}

\maketitle

其实 $S(x)$ 就是平常说的 $d(x)$\ldots{} 只是变量重名不爽所以换了个名字.

首先, 显然有:

\begin{align*}
S(ij) &= \sum_{x|i} \sum_{y|j} [(x, \frac{j}{y}) = 1] \\
      &= \sum_{x|i} \sum_{y|j} [(x, y) = 1]
\end{align*}

特殊地, 当 $(i,j) = 1$ 时, $S(ij) = S(i) \times S(j)$.
这可以说明 $S(x)$ 是一个积性函数.

于是就可以推式子了:

\begin{align*}
F(N,M) &= \sum_{i=1}^{N} \sum_{j=1}^{M} S(i^2) S(j^2) S(ij) \\
       &= \sum_{i=1}^{N} \sum_{j=1}^{M} S(i^2) S(j^2) \sum_{x|i} \sum_{y|j} [(x, y) = 1] \\
       &= \sum_{i=1}^{N} \sum_{j=1}^{M} S(i^2) S(j^2) \sum_{x|i} \sum_{y|j} \sum_{d|x, d|y} \mu(d) \\
       &= \sum_{i=1}^{N} \sum_{j=1}^{M} S(i^2) S(j^2) \sum_{d|i, d|j} \sum_{x|\frac{i}{d}} \sum_{y|\frac{j}{d}} \mu(d) \\
       &= \sum_{i=1}^{N} \sum_{j=1}^{M} S(i^2) S(j^2) \sum_{d|i, d|j} S(\frac{i}{d}) S(\frac{j}{d}) \mu(d) \\
       &= \sum_{d=1}^{\text{min}(N,M)} \sum_{i=1}^{\lfloor\frac{N}{d}\rfloor} \sum_{j=1}^{\lfloor\frac{M}{d}\rfloor} S(i^2d^2) S(j^2d^2) S(i) S(j) \mu(d) \\
       &= \sum_{d=1}^{\text{min}(N,M)} \mu(d) \left(\sum_{i=1}^{\lfloor\frac{N}{d}\rfloor} S(i^2d^2) S(i)\right) \left(\sum_{i=1}^{\lfloor\frac{M}{d}\rfloor} S(i^2d^2) S(i)\right)
\end{align*}

由于 $S(x)$ 是积性函数, 那么 $h(x) = S(x^2)$ 也是积性函数.

所以有:

\begin{align*}
F(N,M) &= \sum_{d=1}^{\text{min}(N,M)} \mu(d) \left(\sum_{i=1}^{\lfloor\frac{N}{d}\rfloor} h(id) S(i)\right) \left(\sum_{i=1}^{\lfloor\frac{M}{d}\rfloor} h(id) S(i)\right)
\end{align*}

首先线性筛筛出 $S(x), h(x)$.

暴力求是单次 $O(n \log n)$ 的. 这样就有 $40$ 分了.

可以预处理一下前缀和, 做到预处理 $O(n \log n)$, 单次询问 $O(n)$, 但是需要 $O(n \log n)$ 的辅助空间. $50$ 分到手.

到目前为止还是太水了.

接下来 $10$ 分的部分分, 由于 $N = M$, 任意时刻 $\lfloor\frac{N}{d}\rfloor = \lfloor\frac{M}{d}\rfloor$, 所以可以分块.

接下来 $10$ 分的部分分, 由于 $M$ 唯一, 所以可以离线出所有的答案. 因为对于所有的 $d$, 当 $N$ 从 $1$ 取到 $10^5$ 时, $\lfloor\frac{N}{d}\rfloor$ 一共也只会变化 $2 \times 10^5 \log 2 \times 10^5$ 次, 所以复杂度一定是正确的.

我的做法:

由于当 $d$ 取 $[5 \times 10^3, 2 \times 10^5]$ 时, $\lfloor \frac{N}{d} \rfloor$ 一共只会有 $20$ 种取值, 所以 $N, M$ 组合起来也不过 $400$ 种取值, 不妨全部预处理出来, 当枚举到的 $d$ 在 $[5 \times 10^3, 2 \times 10^5]$ 中时, 用预处理好的答案分块做; 否则暴力做. 暴力做的规模变为了原来的 $\frac{1}{20}$ , 可以接受.

垃圾出题人卡常?

模数是 $2^{30}$, 自然溢.

至于为什么要卡常, 主要是因为卡常之后的暴力跑得太快了, 也是为了避免被水过去.
% Emacs 25.1.1 (Org mode 8.2.10)
\end{document}