\documentclass[UTF8]{article}
\usepackage{ctex}
\usepackage{hyperref}	%用于设置 PDF 的信息
\usepackage{setspace}	%用于设置行间距
\usepackage{listings}	%用于代码高亮
\usepackage{xcolor}		%用于处理颜色
\usepackage{ulem}		%用于各种线
\usepackage{amsmath}	%用于数学公式(如 \begin{align})
\usepackage{amsthm}		%用于数学版式(如 \newtheorem{cmd}{caption})
\usepackage{booktabs}	%用于表格画线
\usepackage{graphicx}	%用于插入图片
\usepackage[top = 0.75in, bottom = 0.75in, left = 0.75in, right = 0.75in]{geometry} %设置页边距

\newcommand \insertauthor {Orange}
\newcommand \insertsubject {学了就会做的炒鸡简单的题}
\newcommand \Problem [2] {\newpage \section{#1 \small{#2.cpp / #2.in / #2.out}}}
\newcommand \Description {\subsection{题目描述}}
\newcommand \Input {\subsection{输入}}
\newcommand \Output {\subsection{输出}}
\newcommand \Sample [1][] {\subsection{样例 #1}}
\newcommand \SampleInput {\subsubsection{输入}}
\newcommand \SampleOutput {\subsubsection{输出}}
\newcommand \Explanation {\subsubsection{解释}}
\newcommand \Constraint {\subsection{数据规模与约定}}
\newcommand \Hint {\subsection{提示}}

\hypersetup{
	pdfauthor = \insertauthor,
	pdftitle = \insertsubject,
	pdfsubject = \insertsubject,
	pdfkeywords = \insertsubject
}

\title{\insertsubject}
\author{\insertauthor}
\date{\today}

\begin{document}
	\maketitle

	\begin{center}
		\setlength{\tabcolsep}{7mm}
		{
			\begin{tabular}{cccc}
				\toprule
					& 前缀和 & 子集的子集 & 农场 \\
					\hline
					题目类型 & 传统型 & 传统型 & 传统型 \\
					源程序文件名 & prefix.cpp & subsubset.cpp & farm.cpp \\
					输入文件名 & prefix.in & subsubset.in & farm.in \\
					输出文件名 & prefix.out & subsubset.out & farm.out \\
					每个测试点时限 & 1 秒 & 2 秒 & 1 秒 \\
					内存限制 & 512 MiB & 512 MiB & 128 MiB \\
					测试点数目 & (子任务) & (子任务) & 10 \\
					每个测试点分值 & (子任务) & (子任务) & 10 \\
					编译选项 & (默认选项) & (默认选项) & (默认选项) \\
					比较答案方式 & (默认方式) & (默认方式) & (默认方式) \\
				\bottomrule
			\end{tabular}
		}
	\end{center}

	\begin{center}
		\begin{Large}
		请仔细阅读本页内容
		\end{Large}
	\end{center}

	\begin{center}
		\begin{Large}
		所有测试均\textbf{不}打开 -std=c++11 编译选项
		\end{Large}
	\end{center}

	\begin{center}
		\begin{Large}
		比较答案的默认方式为全文比较,忽略文末回车,\textbf{但不忽略行末空格}。
		\end{Large}
	\end{center}

	\begin{center}
		\begin{Large}
			满分 500 分,但像往常一样,实际上你只需要得到 300 分就够了。
		\end{Large}
	\end{center}

	\begin{center}
		\begin{Large}
			如果被电脑卡常,可申述。
		\end{Large}
	\end{center}

	\Problem {前缀和} {prefix}
	\Description

	自从 lyc 学了前缀和后,他就一直在想:前缀和这么厉害,能拿来玩一年吗?

	于是他给了你一个看上去要算一年的任务:对序列 $A$ 求 $m$ 次前缀和,
	求出它的第 $n$ 项(规定下标从 $0$ 开始)。
	其中,序列 $A$ 为:

	$$
	A = a_0, 0, 0, 0, \cdots, 0
	$$

	即第 $0$ 个数为 $a_0$,其余数均为 $0$。保证 $a_0$ 为正整数。

	由于答案可能很大,\textbf{所以结果对 $100003$ 取余}。

	\Input

	本题包含多组询问。

	第一行输入一个整数 $T$,表示数据组数。

	对于每组数据,输入一行三个整数 $n$,$m$,$a_0$,含义均在题目描述中提到。

	\Output

	对于每组询问,输出包含一行一个正整数表示答案。

	\Sample
	\SampleInput

	2

	4 2 3

	2 0 1

	\SampleOutput

	15

	0

	\Explanation

	对于第一个询问,一开始,序列为:
	$$
	3, 0, 0, 0, 0
	$$

	求了第一次前缀和后,序列为:
	$$
	3, 3, 3, 3, 3
	$$

	求了第二次前缀和后,序列为:
	$$
	3, 6, 9, 12, 15
	$$

	所以输出 $15$,注意下标从 $0$ 开始。

	\bigskip

	对于第二个询问,答案显然为 $0$。

	\Constraint

	本题采用子任务测试的形式。
	仅当你通过了子任务中\textbf{所有}的测试点,
	你才能得到该子任务对应的分数,否则不得分。

	\bigskip

	对于 $100 \%$ 的数据,$T \le 10^4$,$0 \le a < 10^5 + 3$。

	子任务 1(24 分):$n, m \le 100$。

	子任务 2(5 分):$n = 1$,$m \le 50000$。

	子任务 3(30 分):$n, m \le 50000$,$T = 1$。

	子任务 4(41 分):$n, m \le 50000$。

	子任务 5(100 分):$n, m \le 10^{18}$。

	\Hint

	注意边界。

	\Problem {子集的子集} {subsubset}
	\Description

	有一个大小为 $n$ 的集合 $S$,
	它由 $n$ 个二元组 $(i, x_i) \pod {1 \le i \le n}$ 组成。
	我们用序列 $s_i \pod {1 \le i \le n}$ 来表示属于 $S$ 的二元组 $(i, s_i)$。
	定义一个集合的权值为属于它的二元组的 $x$ 之积,
	即二元组的第二个元素之积。
	求出 $S$ 的所有非空子集的所有非空子集的权值之和。

	\Input

	输入包含两行。第一行输入一个正整数 $n$,
	第二行输入 $n$ 个\textbf{正整数},表示 $s_{1 \sim n}$。

	\Output

	输出包含一行一个整数,表示答案。
	答案对 $998244353 = 119 \times 2^{23} + 1$ 取模。

	\Sample

	\SampleInput

	2

	2 3

	\SampleOutput

	16

	\Explanation

	$S = \{ (1, 2), (2, 3) \}$,
	为了方便,我们省略二元组的第一个元素,记 $S = \{ 2, 3 \}$。

	\bigskip

	$\{ 2, 3 \}$ 的非空子集有:$\{ 2, 3 \}$,$\{ 2 \}$,$\{ 3 \}$。

	$\{ 2, 3 \}$ 的非空子集的权值之和为:$2 \times 3 + 2 + 3 = 11$。

	$\{ 2 \}$ 的非空子集的权值之和为:$2$。

	$\{ 3 \}$ 的非空子集的权值之和为:$3$。

	所以答案为 $16$。

	\Constraint

	本题采用子任务测试的形式。
	仅当你通过了子任务中\textbf{所有}的测试点,
	你才能得到该子任务对应的分数,否则不得分。

	\bigskip

	对于 $100\%$ 的数据,有 $1 \le s_i < 998244353$。

	子任务 $1$($12$ 分):$n \le 8$。

	子任务 $2$($20$ 分):$n \le 15$。

	子任务 $3$($27$ 分):$n \le 5 \times 10^3$,$\forall i, s_i = 1$。

	子任务 $4$($41$ 分):$n \le 5 \times 10^3$。

	子任务 $5$($100$ 分):$n \le 10^5$。

	\Problem {农场} {farm}

	\Description

	lyc 在 CCF 田野上有一个农场,
	农场里有 $n$ 块土地,从左到右呈一字形排列,
	依次编号为 $1 \sim n$。
	为了防止 DB 来偷菜,他要在某些地下安放监控器。
	每块地里最多只能建立一个监控器。
	设土地 $i$ 左边离它最近的第一个监控器的位置在 $d_i$,
	如果在 $i$ 安放监控器,那么它要监控的土地为 $d_i + 1 \sim i$。
	特别地,第一个监控器要监控的土地为 $1 \sim i$。
	由于离监控器越远的土地需要的技术含量越高,
	lyc 对每一块土地的花费等于它到监控它的监控器之间的土地数目
	(不包括自身,但包括监控器所在的土地)\textbf{乘上}该土地的种植量。

	已知在第 $i$ 块土地安放监控器的花费是 $a_i$,
	第 $i$ 块土地的种植量是 $b_i$。
	请你帮 lyc 求出最小总花费。

	\Input

	输入包含 $3$ 行。

	第一行输入一个整数 $n$,表示土地数目。

	第二行输入 $n$ 个整数,第 $i$ 个整数表示 $a_i$。

	第三行输入 $n$ 个整数,第 $i$ 个整数表示 $b_i$。

	\Output

	输出包含一行一个整数,表示答案。

	\Sample

	\SampleInput

	4

	2 4 2 4

	3 1 4 2

	\SampleOutput

	9

	\Explanation

	选取第 $1$,$3$,$4$ 号土地安放监控器,
	此时的费用最小,为 $2 + (2 + 1 \times 1) + 4 = 9$。

	\Constraint

	对于 $10 \%$ 的数据,$n \le 10$。

	对于 $40 \%$ 的数据,$n \le 10^3$。

	对于 $100 \%$ 的数据,$n \le 10^6$,$1 \le a_i, b_i \le 10^4$。

\end{document}