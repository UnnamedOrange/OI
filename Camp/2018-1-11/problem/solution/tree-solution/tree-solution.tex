% Created 2017-01-18 Wed 21:14

\documentclass[12pt,a4paper]{article}
\usepackage[top = 1.0in, bottom = 1.0in, left = 1.0in, right = 1.0in]{geometry}

\usepackage{hyperref}
\hypersetup{
  colorlinks=true,
  linkcolor=[rgb]{0,0.37,0.53},
  citecolor=[rgb]{0,0.47,0.68},
  filecolor=[rgb]{0,0.37,0.53},
  urlcolor=[rgb]{0,0.37,0.53},
  pagebackref=true,
  linktoc=all,}

\usepackage{hyperref}
\usepackage[utf8]{inputenc}
\usepackage{ctex}
\usepackage{fixltx2e}
\usepackage{graphicx}
\usepackage{longtable}
\usepackage{float}
\usepackage{wrapfig}
\usepackage{rotating}
\usepackage[normalem]{ulem}
\usepackage{amsmath}
\usepackage{textcomp}
\usepackage{marvosym}
\usepackage{wasysym}
\usepackage{multicol}
\usepackage{amssymb}
\tolerance=1000
\author{h10}
\date{\today}
\title{Tree Solution}
\hypersetup{
  pdfkeywords={},
  pdfsubject={},
  pdfcreator={Emacs 25.1.1 (Org mode 8.2.10)}}
\begin{document}

\maketitle

大家知道什么叫做暴力美学吗?
就是用暴力来 A 一道题.

没错这题的核心思想就是暴力.
维护 $[2, n]$ 的父亲, 注意当一整个区间的父亲都是一样的时候, 我们将这个区间合并起来.

如当 \texttt{fa[]} 数组为 \texttt{fa[2]=1}, \texttt{fa[3]=1} , \texttt{fa[4]=1}, \texttt{fa[5]=4}, \texttt{fa[6]=4}, \texttt{fa[7]=2} , \texttt{fa[8]=2} 时, 我们维护的东西长这样
$$[2,3,4]\;[5,6]\;[7,8]$$

如果我们将 $[3,7]$ 的父亲赋值为 1, 将变成 
$$[2]\;[3,4,5,6,7]\;[8]$$

注意每次操作最多会切开两个断点, 每个断点处理一次就会消失, 所以哪怕暴力维护也只用搞 $O(n)$ 次.

考虑用 Splay 维护树的括号序列, 于是每次操作就变成了将 Splay 上连续的一段移动到另一个位置.
但是维护的时候要小心, 如对于这个括号序列 
\begin{verbatim}
((())()) 
12332441
\end{verbatim}

如果我们将 4 的父亲赋值为 2 
\begin{verbatim}
((()())) 
12443321
\end{verbatim}

这样就错了.
3 与 4 的顺序搞反了.

在这组数据里没有什么问题, 数据大一点就会出奇怪的问题.
正确的是这样
\begin{verbatim}
((()())) 
12334421
\end{verbatim}

要是你问我怎么避免, 我只能说无可奉告.
对每个点开个 \texttt{set} 记录它的儿子就好了.
当然 \texttt{set} 里的东西还得要能合并就合并.
% Emacs 25.1.1 (Org mode 8.2.10)
\end{document}
